%%
%% Package: spectralsequences v1.0.0-dev
%% Author: Hood Chatham
%% Email: hood@mit.edu
%% Date: 2017-06-21
%% License: Latex Project Public License
%%
%% File: example_ass.tex
%%
%%    Draws the Adams Spectral Sequence at the prime 2 through the 45 stem.
%%    See pages 146 - 147 of Green book
%%    https://mathoverflow.net/questions/102316/differentials-in-the-adams-spectral-sequence-for-spheres-at-the-prime-p-2
%%    https://en.wikipedia.org/wiki/Homotopy_groups_of_spheres#Table_of_stable_homotopy_groups
%%


\documentclass{article}
\usepackage[landscape,paperheight=15in,margin=0.1in]{geometry}
\usepackage{spectralsequences}


\begin{document}
\sseqset{imJ/.sseq style={purple}}


\NewSseqCommand\twoclass{O{}d()}{
    \IfNoValueF{#2}{\pushstack(#2)}
    \class[#1](\lastx,\lasty+1)
    \structline(\lastclass1)(\lastclass)
}

\NewSseqCommand\divtwoclass{O{}d()}{
    \IfNoValueF{#2}{\pushstack(#2)}
    \class[#1](\lastx,\lasty-1)
    \structline(\lastclass1)(\lastclass)
}


\NewSseqCommand\etaclass{O{}d()}{
    \IfNoValueF{#2}{\pushstack(#2)}
    \class[#1](\lastx+1,\lasty+1)
    \structline(\lastclass1)(\lastclass)
}
\NewSseqCommand\divetaclass{O{}d()}{
    \IfNoValueF{#2}{\pushstack(#2)}
    \class[#1](\lastx-1,\lasty-1)
    \structline(\lastclass1)(\lastclass)
}
\NewSseqCommand\etastruct{d()}{
    \IfNoValueF{#1}{\pushstack(#1)}
    \structline(\lastx-1,\lasty-1,-1)(\lastclass)
}


\NewSseqCommand\nuclass{O{}d()}{
    \IfNoValueF{#2}{\pushstack(#2)}
    \class[#1](\lastx+3,\lasty+1)
    \structline(\lastclass1)(\lastclass)
}

\NewSseqCommand\nustruct{d()}{
    \IfNoValueF{#1}{\pushstack(#1)}
    \structline(\lastx-3,\lasty-1,-1)(\lastclass)
}

\NewSseqCommand\hznudiamond{d()}{
    \IfNoValueF{#1}{\pushstack(#1)}
    \savestack
    \class(\lastx,\lasty+1)
    \structline(\lastclass1)(\lastclass)
    \nustruct
    \restorestack
}

\NewSseqCommand\etanudiamond{d()}{
    \IfNoValueF{#1}{\pushstack(#1)}
    \savestack
    \class(\lastx+1,\lasty+1)
    \structline(\lastclass1)(\lastclass)
    \nustruct
    \restorestack
}

\NewSseqCommand \czero {O{} r() }{
    \savestack
    \class[#1](#2)
    \etaclass
    \restorestack
}

\NewSseqCommand\tower {O{}d()mO{}} {
    \begin{scope}[#4]
    \IfNoValueF{#2}{\class[#1](#2)}
    \savestack
    \Do{#3-1}{
        \class(\lastx, \lasty+1)
        \structline(\lastclass)(\lastclass1)
    }
    \restorestack
    \end{scope}
}


\NewSseqCommand \towergroup {o r() m O{}} {
    \tower(#2){#3}[#4]
    \IfNoValueF{#1}{\classoptions[#1](#2)}
    \etaclass\etaclass\etaclass
    \nustruct\divtwoclass\nustruct\divtwoclass
    \nustruct\nuclass
}


\protected\def\Psqvphant{\vphantom{P^2}}
\NewSseqGroup \Phone { mm } {
    \class["#1h_1#2"](0,0)
    \etaclass\etaclass
    \divtwoclass\divtwoclass["\Psqvphant#1h_2#2" {below=-4pt}]
}

\NewSseqGroup \Phonegroup {G{}}{
    \sseqset{class label handler = { \sseqnormalizemonomial{##1} }}
    \czero["#1c_0" above](-1,-2)
    \Phone[imJ]{P#1}{}
    \nuclass[imJ]\divtwoclass[imJ]\divtwoclass["#1d_0" left,name=d0]
    \etaclass\etaclass\etaclass
    \nustruct\divtwoclass\nustruct\divtwoclass\nustruct\divtwoclass["#1e_0" left,name=e0]
    \nuclass\hznudiamond
    \divtwoclass["#1g" {below = 0.3em}, name=g]
    %\etanudiamond

    % Have to go back and draw path e0 to g
    \savestack
    \d2(e0,-1)
    \etaclass(e0)\d2(\lastclass,-1)
    \divtwoclass[name=f0]\d2(\lastclass,-1)
    \nuclass[name=nuf0]
    \etastruct % Now we're back at g
    \restorestack
    \nuclass[name=nug]\hznudiamond\nuclass[name=nu2g]
}


\NewSseqGroup \Phonegroupb {G{}}{
    \Phonegroup{P#1}
    
    \divtwoclass["\Psqvphant#1j" {below left=-1pt,yshift=2pt}](f0) \d2(\lastclass,-1)
    \divtwoclass(nuf0) \d2(\lastclass,-1)
    \nustruct\divtwoclass["\Psqvphant#1k" {below=-1pt}] \d2(\lastclass,-1)
    \nuclass\d2(\lastclass,-1)
    \hznudiamond
    \divtwoclass["\Psqvphant#1l" {below left=-1pt,yshift=2pt}]\d2(\lastclass,-1)
    \nuclass\d2(\lastclass,-1)\hznudiamond
    \divtwoclass["\Psqvphant#1m" {below left=-3pt},name=m]

    \divtwoclass(nug)\structline(\lastclass)(\lastx+1,\lasty+1)
    \divtwoclass(nu2g)\nustruct
    \divtwoclass\structline(\lastclass)(\lastx+1,\lasty+1)
    \d2(m,-1)
}



\begin{sseqdata}[
    name = S0ASS,
    Adams grading,
    classes=fill,
    class labels={above left=0.2em,black},
    x range={0}{46},
    y range={0}{23},
    xscale=0.7,
    yscale=0.8,
    differentials = blue,
    classes = { tooltip = { (\xcoord,\ycoord) } },
    grid = go,
    run off = -
]
\towergroup(0,0){\ymax-\ymin+2}
\classoptions["h_2" below](3,1)
\nuclass
\divetaclass\divetaclass["h_3" below]
\tower{4}


\Phonegroup(9,5)
\classoptions["f_0"](f0)

\Phonegroupb(17,9)
\pushstack(d0)
\tower["i" {below left,yshift=2pt}](23,7){6}[imJ]
\structline(\lastclass)(\lastx+3,\lasty+1,-1)
\structline(\lastx,\lasty+1)(\lastx+3,\lasty+2,-1)
\d2(23,7)
\d2(23,8)
    
\Phonegroupb(25,13){P}
\Phonegroupb(33,17){P^2}
\Phonegroup(41,21){P^4}


\towergroup["h_4" below](15,1){8}[imJ]
\classoptions["h_3^3" below](\lastclass)

\tower["h_3^2" left](14,2){2}
\d2(15,1) % d2(h_4) = h_0h_3^2
\d3(15,2)\d3(15,3) % d2(h_0h_4) = h_0d_0

\class["c_1"](19,3)\nuclass
\czero["h_4c_0"](23,4)



\towergroup["h_5" below](31,1){16}[imJ]
\nuclass
\divetaclass\divetaclass["h_3h_5" {below=-2pt}]
\tower{4}

\class["n"{above=-0.5pt}](31,5)
\nuclass\nuclass
\divetaclass["t"]
\class["q" {below right=-0.7pt}](32,6)
\etaclass

\class["p"](33,4)
\twoclass\divetaclass["d_1"]
\nuclass\nuclass
\divetaclass["x"]\tower{6}

\d3(38,4)
\d4(38,3,,2)
\d4(38,2)


\tower["h_4^2" below](30,2){4}
\etaclass
\tower["r" left](30,6){6}

\d2(31,1)\d2(31,2)\d2(31,3)
\d3(31,4)\d3(31,5)\d3(31,6)\d3(31,7)\d3(31,8,-1)
\d3(31,9,-1,-1)\d3(31,10,-1,-1)


\class(38,6)\d2(\lastclass,-1)
\twoclass\d2(\lastclass,-1)
\twoclass\nustruct\d2(\lastclass,-1)
\twoclass\nustruct\d4(\lastclass,-1)
\divetaclass["e_0g"]\d4(\lastclass,-1)

\czero[pin distance=3em,pin=below right:h_5c_0"](39,4)

\class["e_1" {below=-1pt}](38,4)
\etaclass\etaclass\divtwoclass\divtwoclass["f_1" below]
\etaclass
\divtwoclass\d2(\lastclass)
\divtwoclass\d2(\lastclass)


\Phone(40,6){P}{h_5}
\class["c_1g"](39,7)

\class["u"](39,9)\etaclass\etaclass\divtwoclass

\class["g^2"](40,8)
\class["v"](42,9)

\tower["P^2r" {below=-4pt}](46,14){6}
\d3(\lastclass)

\class["Pe_0g" right](45,12)
\d4(\lastclass)

\class["w"](45,9)

\class["d_0r" right](44,10)
\d3(\lastclass)

\tower["g_2" below](44,4){3}
\etaclass

\class["h_4^3" below](45,3)\twoclass

\tower["h_5d_0" {below right=-2pt}](45,5){3}
\etaclass\etaclass
\nuclass(45,5)\hznudiamond

\class["N"](46,8)
\class["gj"](46,11)

\tower(47,13){12}
\foreach \y in {13,...,18} { \d2(47,\y)}
\end{sseqdata}
\centering
\printpage[name=S0ASS,page=0]

\printpage[name=S0ASS,page=2]

\printpage[name=S0ASS,page=3]

\printpage[name=S0ASS,page=4]

\begin{sseqpage}[name=S0ASS,page=5]
\structline[dashed](14,2)(15,5,1)
\structline[dashed](21,5)(22,8)
\structline[dashed,bend right=20](20,6)(23,9,-1)
\structline[dashed](23,6)(23,9,-1)
\structline[dashed,bend right=20](30,2)(33,4)
\structline[dashed](40,8)(40,10)
\structline[dashed](40,8)(41,10)
\structline[dashed,bend left=40](39,9)(42,12)
\structline[dashed](41,10)(42,12)
%\structline[dashed,bend right=20](32,6)(33,9)

\classoptions["h_0^2i" left](23,9)
\classoptions["h_0^{10}h_5" left](31,11)
\classoptions["h_1h_5" below](32,2)
\classoptions["h_2h_5" below](34,2)
%\classoptions["h_0^2h_3h_5" {left=-0.1em}](38,4,1)
\classoptions["h_1h_3h_5" {below=-.4em}](39,3,1)
\classoptions["P^4h_3" {below=0em}](39,17)
\end{sseqpage}
\end{document} 