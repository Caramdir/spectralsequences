%%
%% Package: spectralsequences v1.0.0-dev
%% Author: Hood Chatham
%% Email: hood@mit.edu
%% Date: 2017-06-21
%% License: Latex Project Public License
%%
%% File: example_anss3.tex
%%
%%    Draws the Adams Novikov spectral sequence at the prime 3 through the 45 stem.
%%    In this range, all we see is the Toda differential d_{2p-1}(\beta_{p/p}) = \alpha\beta^p
%%    Thanks to Eric Peterson for contributing this diagram. Presumably he copied it from page 13 of the green book.
%%    Compare this to page 5 of example_ass3
%%

\documentclass{article}
\usepackage{spectralsequences}
% 8.5 x 15 is pretty much a 16/9 aspect ratio, ideal for on-screen viewing
\usepackage[landscape, paperheight=15in,margin=0.1in, top=1in]{geometry}
\begin{document}
% tower definitions
\NewSseqCommand\alphaclass{d()}{
    \IfNoValueTF{#1}{}{\pushstack(#1)}
    \class(\lastx+3,\lasty+1)
    \structline(\lastclass)(\lastclass1)
}

\NewSseqCommand\betaclass{d()}{
    \IfNoValueTF{#1}{}{\pushstack(#1)}
    \class(\lastx+7,\lasty+1)
    \structline(\lastclass)(\lastclass1)
}

\centering
\begin{sseqpage}[
    Adams grading,
    classes = { tooltip = {(\xcoord,\ycoord)}, inner sep=1.2pt },
    class labels = above left,
    label distance=3pt,
    differentials={-{>[width=4]}, target anchor=-60},
    y range={0}{11},
    x range={0}{45},
    x tick step=5,
    xscale=0.7,
    yscale=1.2,
    y axis gap=2em
]
\class[rectangle,fill,inner sep=3pt](0,0)

\class["\alpha_1"](3,1) \structline(0,0)(3,1)
\class["\alpha_2"](7,1)
\class[circlen=2,"\alpha_{3/2}"](11,1)
\class["\alpha_4"](15,1)
\class["\alpha_5"](19,1)
\class[circlen=2,"\alpha_{6/2}"](23,1)
\class["\alpha_7"](27,1)
\class["\alpha_8"](31,1)
\class[circlen=3,"\alpha_{9/3}"](35,1)
\class["\alpha_{10}"](39,1)
\class["\alpha_{11}"](43,1)

\class["\beta_1"](10,2) \structline(3,1)(10,2)
\class["\beta_2"](26,2)
\class["\beta_{3/3}" {xshift = -2pt, yshift=-2pt}](34,2)
\class["\beta_{3/2}"](38,2)
\class["\beta_3"](42,2)

% tower off of beta_1
\alphaclass(10,2)\betaclass
\alphaclass\betaclass
\alphaclass\betaclass
\alphaclass\betaclass

% tower off of beta_2
\alphaclass(26,2)\betaclass
\alphaclass\betaclass
\alphaclass


% tower off of beta_3/3
\alphaclass(34,2)\betaclass
\alphaclass\betaclass

% tower off of beta_3
\class[fill,circlen=2](45,3) \structline(42,2,-1)(45,3,-1)
\betaclass

% d5s
\d5(34,2)
\d5(44,4)
\end{sseqpage}

\end{document} 