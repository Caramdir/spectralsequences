%%
%% Package: spectralsequences v1.0.0-dev
%% Author: Hood Chatham
%% Email: hood@mit.edu
%% Date: 2017-06-21
%% License: Latex Project Public License
%%
%% File: example_hatcher.tex
%%
%%    This comes from Hatcher's spectral sequences book. I think it's a good demonstration of the flexibility of sseqpages.
%%    Thanks to Antonio Ruiz for suggesting it.
%%

\documentclass{article}
\usepackage[landscape,margin=1cm,top=2cm]{geometry}
\usepackage{spectralsequences}
\usepackage{amssymb}
\def\Z{\mathbb{Z}}

\begin{document}
\sseqset{
    0/.sseq style={"0",draw=none},
    homological Serre grading,
    classes={draw=none,inner sep=0.2em},
    tick gap=0.7em,
}
\begin{sseqdata}[
    name=hatcherex1,
    permanent cycles={draw,minimum width={width("$Z_2$")+0.85em}},
    yscale=0.6,
    axes gap=1.2em,
    axes clip padding=0em
]
\begin{scope}[background,opacity=0.1]
\foreach \n in {1,3,...,9}{
    \fill(-1.1,\n-0.5)--(-1.1,\n+0.5)--(0,\n+0.5)--(\n+0.5,0)--(\n+0.5,-1.6)--(\n-0.5,-1.6)--(\n-0.5,0)--(0,\n-0.5)--cycle;
}
\end{scope}

\class["\Z"](0,0)
\foreach\x in {1,3,...,9}{
    \class["\Z_2"](\x,0)
}
\foreach \x in {2,4,...,8}{
    \class[0](\x,0)
}
\foreach\y in {1,3,...,9}{
    \pgfmathparse{9-\y}
    \let\xmax\pgfmathresult
    \foreach\x in {0,...,\xmax}{
        \class["\Z_2"](\x,\y)
    }
}

\foreach \y in {2,4,...,8}{
    \pgfmathparse{9-\y}
    \let\xmax\pgfmathresult
    \foreach\x in {0,...,\xmax}{
        \class[0](\x,\y)
    }
}

\foreach \x in {3,5,...,9}{
    \d2(\x,0)
}

\d3(4,1)
\d3(6,1)
\d3(8,1)
\d3(4,3)
\d3(6,3)
\d3(4,5)
\end{sseqdata}
\printpage[name=hatcherex1,page=0]

\begin{sseqdata}[
    name=hatcherex2,
    yscale=0.6,
    x axis gap=0.3cm,
]

\begin{scope}[background,opacity=0.1]
\foreach \n in {1,3,...,9}{
    \fill(-1.1,\n-0.5)--(-1.1,\n+0.7)--(0,\n+0.7)--(\n+0.5,0)--(\n+0.5,-1.6)--(\n-0.5,-1.6)--(\n-0.5,0)--(0,\n-0.5)--cycle;
}
\end{scope}


\foreach\x in {0,2,...,8}{
    \class["\Z"](\x,0)
}
\foreach \x in {1,3,...,9}{
    \class[0](\x,0)
}
\foreach\y in {1,3,...,9}{
    \pgfmathparse{9-\y}
    \let\xmax\pgfmathresult
    \foreach\x in {0,...,\xmax}{
        \pgfmathparse{int(mod(\x,2))}
        \ifnum\pgfmathresult=0\relax
            \class["\Z_2"](\x,\y)
        \else
            \class[0](\x,\y)
        \fi
    }
}

\d2(2,0)
\foreach \x in {4,6,...,8}{
    \foreach\r in {2,4,...,\x}{
        \d\r(\x,0)
        \replaceclass["\Z"](\x,0)
    }
}

\foreach \y in {2,4,...,8}{
    \pgfmathparse{9-\y}
    \let\xmax\pgfmathresult
    \foreach\x in {0,...,\xmax}{
        \class[0](\x,\y)
    }
}

\end{sseqdata}
\printpage[name=hatcherex2,page=0]
\end{document}
