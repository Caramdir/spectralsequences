%%
%% Package: spectralsequences v1.0.0-dev
%% Author: Hood Chatham
%% Email: hood@mit.edu
%% Date: 2017-06-21
%% License: Latex Project Public License
%%
%% File: example_KUHFPSS.tex
%%
%%    I think this is one of Eric Peterson's favorite spectral sequences, and his enthusiasm for it has rubbed off on me a bit.
%%    Here our group is C_2 which acts on the periodicity element by -1, so again we have a norm element v in degree 2|G| = 4.
%%    There are easier ways to understand this, but I explained the EO_3 one in terms of the comparison map from the ANSS,
%%    and that's interesting here too. In the ANSS at 2, there is a differential d3(\alpha_3) = \alpha_1^4. Here a cobar calculation
%%    shows that \alpha v =  \alpha_3, so dividing the differential by \alpha gives d3(v) = \alpha_1^3. Now there's no Kudo differential
%%    because the prime is 2, and the spectral sequence immediately collapses.
%%
%%    Second, a demonstration of the falsehood of ku^{hC_2} = ko  --  ku^{hC_2} has an extra generator as a ring, which is in degree -4. 
%%    In particular, it's not even connective.
%%

\documentclass{article}
\usepackage[landscape,margin=2cm]{geometry}
\usepackage{spectralsequences}
\begin{document}
\sseqset{
    Z2class/.sseq style={circle,inner sep=0.3ex,fill=black},
    Zclass/.sseq style={fill=none,draw,inner sep=0.6ex},
    2Zclass/.sseq style={fill=none,rectangle,draw,inner sep=0.6ex,outer sep=0.5ex}
}
\begin{sseqdata}[
    name=KRHFPSS,
    x range={-12}{14},
    y range={0}{10},
    y axis style=center,
    y axis gap=0.425cm,
    tick step=4,
    classes=Z2class,
    differentials=->,
    degree={-1}{#1-1},
    scale=0.85,
    right clip padding=0.1cm,
    top clip padding=0.05cm,
    x axis extend start=0cm,
    x axis extend end=0.33cm,
    y axis extend end=0.3cm,
]

\draw[background,xshift=-0.5cm,yshift=-0.51cm,step=1cm,gray,very thin] (\xmin+0.01,\ymin+0.01) grid (\xmax+0.9,\ymax+0.9);

\pgfmathsetmacro\xitstart{int(int(\xmin/8)*8-16)}
\pgfmathsetmacro\xitgap{int(\xitstart+4)}
\pgfmathsetmacro\xitend{int(\xmax+2)}
\pgfmathsetmacro\xmaxpp{int(\xmax+2)}

\foreach \x in {\xitstart,\xitgap,...,\xitend} {
    \class[Zclass](\x,0)
    \foreach \z in {0,...,\xmaxpp} {
        \class(\x+\z+1,\z+1)
        \structline(\x+\z,\z)(\x+\z+1,\z+1)
    }
}

\pgfmathsetmacro\xitstart{int(\xitgap)}
\pgfmathsetmacro\xitgap{int(\xitstart+8)}

\foreach \x in {\xitstart,\xitgap,...,\xitend} {
    \foreach\z in {0,...,\xmax}{
        \d4(\x+\z,\z)
    }
    \replaceclass[2Zclass](\x,0)
}
\end{sseqdata}
\printpage[name=KRHFPSS,page=0]
\newpage
\printpage[name=KRHFPSS,page=5]


\newpage
\begin{sseqpage}[name=KRHFPSS,page=0,keep changes]
\pgfmathsetmacro\antidiag{min(-\xmin,\ymax+0.8)}
\clip[background,xshift=0.28cm,yshift=-0.4cm](-\antidiag,\antidiag)--(-1,1)--(-0.4,0)--(\xmax+2,0)--(\xmax+2,\antidiag)--cycle;
\foreach \z in {2,6}{
    \doptions[draw=none]4(-\z-1,\z-1)
    \structlineoptions[draw=none](-\z-1,\z-1)(-\z,\z)
    \structlineoptions[draw=none](-\z-3,\z+1)(-\z-2,\z+2)
    \replaceclass(-\z-2,\z+2)
}
\structlineoptions[draw=none](-3,1)(-2,2)
\end{sseqpage}
\newpage
\printpage[name=KRHFPSS,page=5]
\end{document} 