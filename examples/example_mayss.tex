%%
%% Package: spectralsequences v1.0.0-dev
%% Author: Hood Chatham
%% Email: hood@mit.edu
%% Date: 2017-06-21
%% License: Latex Project Public License
%%
%% File: example_mayss.tex
%%
%%    I think this is the May SS for the sphere. May spectral sequences are annoying to draw neatly it turns out -- there's too much stuff in them.
%%    It helps a lot if you start at E_2 though, since E_1 is polynomial. They also have the problem with their crappy grading that makes all
%%    differentials the same length. I seem to have graded this weirdly, but I have no idea why.
%%

\documentclass{article}
\usepackage{fullpage}
\usepackage{spectralsequences}

\begin{document}
\NewSseqGroup\tower {m} {
    \class["#1"](0,0)
    \foreach\i in {1,...,14}{
        \class(0,\i)
        \structline(0,\i-1,-1)(0,\i,-1)
    }
}

\NewSseqGroup\hone {m} {
    \foreach\i in {1,...,#1}{
        \class(\i,\i)
        \structline(\i-1,\i-1,-1)(\i,\i,-1)
    }
}

\NewSseqCommand\dtower {u(u)} {
    \foreach\i in {0,...,10}{
        \d[yshift=\i]#1(#2)
    }
}


\def\single(#1)#2{\class["#2"](#1)}

\NewSseqGroup\htwo {m} {
    \foreach \n in {0,...,5}{
        \tower(3*\n,\n){\sseqifempty{#1}{}{#1}\sseqpowerempty{h_2}{\n}}
        \ifnum\n>0\relax
            \structline(3*\n-3,\n-1,-1)(3*\n,\n,-1)
        \fi
    }
}

\NewSseqGroup\htwosinglejoin {mm} {
    \foreach \n in {0,...,5}{
        \single(3*\n,\n){\sseqifempty{#1}{}{#1}\sseqpowerempty{h_2}{\n}}
        \ifnum\n>0\relax
            \structline(3*\n-3,\n-1,-1)(3*\n,\n,-1)
        \fi
    }
    \foreach \n in {0,...,#2}{
        \structline(3*\n,\n,-1)(3*\n,\n+1,-1)
    }
}



\begin{sseqdata}[name=may, degree={-1}{1-#1},x range={0}{13}, y range={0}{13},class labels=below right,differentials=blue,x axis extend end=23pt,draw]
\tower(7,1){h_3}
\htwo(0,0){}
\classoptions[label position=above left](9,3)
\hone(0,0){3}
\htwo[label position={xshift=3pt,yshift=2pt}](4,4){b_{2,0}}
\dtower2(4,4,-1,-1) % d2(b_{2,0}h_0^n) = h_2h_0^{n+2}
\d2(4,4,-1,1) % d2(b_{2,0})= h_2h0^2 + h1^3

%\hone(4,4){3}
\htwo(8,8){b_{2,0}^2}
\hone(8,8){3}
\dtower4(8,8,,1) % d4(b_{2,0}^2) = h3 h0^4
%\hone(8,8){3}
\hone(7,1){2}


\htwosinglejoin(7,4){x_7}{4}
\dtower2(7,4,2,-1)
\hone(7,4){2}
\htwosinglejoin(11,8){x_7b_{2,0}}{4}
\htwosinglejoin(10,4){b_{2,1}}{3}
%\hone(10,4){1}
\d2(10,4,-1,2) % d2(b21)=h2^3+h3h1^2
\dtower2(10,4,1,1) % d2(b21 h0^n) = h2^3 h0^n


\tower(11,5){h_3b_{2,0}}
\tower(12,6){b_{3,0}}
\dtower2(12,6,-1,-1)

\htwo(12,12){b_{2,0}^3}
\dtower2(12,12,-1,1) % d2(b20^3 h_0^n) = h0^n b20^2 d2(b20) = h0^{n+2} h2 b20^2
\d2(12,12,-1,2) % d2(b_{2,0}^3)= b20^2 d2(b20) = h_2 h0^2 b20^2 + h1^3 b20^2
\hone(12,12){3}
\replaceclass(11,11)

\structline(11,10)(11,11)
\structline(10,10,2)(11,11)

\replaceclass(3,3)

\structline(2,2)(3,3)
\structline(3,2)(3,3)





\replaceclass(9,3)
\structline(6,2)(9,3)
\structline(8,2)(9,3)



%\dtower2(12,6,2,-1)
\end{sseqdata}
\printpage[name=may]
\newpage
\printpage[name=may, page=4]
\newpage
\printpage[name=may, page=5]
\end{document} 