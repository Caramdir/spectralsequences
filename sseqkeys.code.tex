%%
%% Package: spectralsequences v1.0.0-dev
%% Author: Hood Chatham
%% Email: hood@mit.edu
%% Date: 2017-06-21
%% License: Latex Project Public License
%%
%% File: sseqkeys.code.tex
%% Exposes: \sseqset
%%
%%       Sets up the keys for sseqpages. There are a lot of them...
%%


% Run \pgfqkeys{directory}{keys} but for a list of directories. Helps to avoid code duplication.
\def\sseq@pgfqkeysdirlist#1#2{%
    \sseq@temptoks{#2}%
    \sseq@pgfqkeysdirlist@#1,\sseq@nil
}

\def\sseq@pgfqkeysdirlist@#1,{%
    \sseq@eval{\@nx\pgfqkeys{#1}{\the\sseq@temptoks}}%
    \@ifnextchar\sseq@nil{\@gobble}{\sseq@pgfqkeysdirlist@}%
}

%%%% Interface macros:
% These are the things that are used outside of this section:
\let\sseq@pgfkeys@error@save\pgfkeys@error
\pgfkeys{/sseqpages/userstyles/.unknown/.code={\sseq@passoptionto{\sseq@sseqsetdirectory}}}
\protected\def\sseqset#1{\pgfqkeys{/sseqpages/userstyles}{#1}}

\def\sseq@sseqsetdirectory{/sseqpages/global/default}

\def\sseq@set{\pgfqkeys{/sseqpages}}

%%%
%%%   Local Keys
%%%

% This is where and how we are going to store the results of our local options. Note that definitions need to be local to allow recursion.
\def\sseq@savedoptioncode{}
\def\sseq@d@addto@options{\sseq@d@addto@macro\sseq@savedoptioncode}
\def\sseq@e@addto@options{\sseq@e@addto@macro\sseq@savedoptioncode}

% #1 -- name of key
% #2 -- options
% This is the main way that we use options. (We also use \pgfqkeys{/sseqpages/global} and some similar things)
% Dumps out code that the option indexes into \sseq@savedoptioncode
% This is because \pgfkeys has a high performance cost, so instead of using \pgfkeys once per page,
% we use this once when putting together the spectral sequence and save the result.
\def\sseq@processoptions#1#2{%
    \def\sseq@savedoptioncode{}%
    \sseq@protectedeval{\@nx % TODO: get rid of this protectedeval
        \pgfqkeys{/sseqpages/#1}{#2}
    }%
}

% First pass versus second pass:
% Lots of options mainly do stuff on the drawing run, but still need to warn the sseq setup algorithms that they're around
% on the first pass. Some only do things on the first pass or only do things on the second pass.
% For example, shifts: shifts apply a coordinate transform on the second pass, but also need to factor into the coordinates of
% classes etc which are computed at specification time not at draw time. So xshift=2 increases \sseq@x by 2 on it's first pass and
% applies a coordinate transform by 2cm on the second pass.

% normal mode does the firstpass options now and saves the secondpass options in \sseq@savedoptioncode for later use.
% firstpass mode does the firstpass options and throws away the secondpass options.
% secondpass mode throws away the firstpass options and does the secondpass options NOW
% bothpass mode does both firstpass and secondpass options NOW
% We spend the vast majority of our time in normal mode.

\def\sseq@options@use#1{#1}
\def\sseq@options@normalmode{\let\sseq@options@firstpass\sseq@options@use\let\sseq@options@secondpass\sseq@d@addto@options}
\def\sseq@options@firstpassmode{\let\sseq@options@firstpass\sseq@options@use\let\sseq@options@secondpass\@gobble}
\def\sseq@options@secondpassmode{\let\sseq@options@firstpass\@gobble\let\sseq@options@secondpass\sseq@options@use}
\def\sseq@options@bothpassmode{\let\sseq@options@firstpass\sseq@options@use\let\sseq@options@secondpass\sseq@options@use}
\sseq@options@normalmode

\pgfqkeys{/handlers}{
    .sseq @ first pass code/.code={
        \pgfkeysalso{\pgfkeyscurrentpath/.code={\sseq@options@firstpass{#1}}}
    },
    .sseq @ first pass append code/.code={
        \pgfkeysalso{\pgfkeyscurrentpath/.append code={\sseq@options@firstpass{#1}}}
    },
    .sseq @ second pass code/.code={
        \pgfkeysalso{\pgfkeyscurrentpath/.code={\sseq@options@secondpass{#1}}}
    },
    .sseq @ second pass append code/.code={
        \pgfkeysalso{\pgfkeyscurrentpath/.append code={\sseq@options@secondpass{#1}}}
    },
    .sseq @ second pass store in/.code={
        \pgfkeysalso{\pgfkeyscurrentpath/.code={\sseq@options@secondpass{\def#1{##1}}}}
    },
    .sseq @ code/.code 2 args={
        \pgfkeysalso{\pgfkeyscurrentpath/.code={\sseq@options@firstpass{#1}}}
        \pgfkeysalso{\pgfkeyscurrentpath/.append code={\sseq@options@secondpass{#1}}}
    },
    .sseq @ transform/.code={
        \sseq@keys@lastpart
        \edef\sseq@temppath{\pgfkeyscurrentpath/\sseq@temp}
        \pgfkeysalso{\sseq@temppath/.code/.expanded={\@nx\sseq@checktransform{\sseq@temp}\unexpanded{{##1}}}}
        \pgfkeysalso{\sseq@temppath/.sseq @ second pass append code/.expanded={\@nx\sseq@pgfkeysdocommand{\sseq@temp}\unexpanded{{##1}}}}%
    },
    .sseq @ stdparse do transform/.code={
        \sseq@keys@lastpart
        \edef\sseq@temppath{\pgfkeyscurrentpath/\sseq@temp}
        \ifx\pgfkeyscurrentvalue\pgfkeysnovalue@text
            \pgfkeysalso{\sseq@temppath/.code/.expanded={\@nx\sseq@standardparse\@nx\sseq@options@secondpass{\@nx\sseq@pgfkeysdocommand{\sseq@temp}\unexpanded{{##1}}}}}
        \else
            \pgfkeysalso{\sseq@temppath/.code/.expanded={\@nx\sseq@standardparse\@nx\sseq@options@secondpass{\@nx\sseq@pgfkeysdocommand{\sseq@temp}{#1}}}s}
        \fi
    },
    .sseq @ illegal/.code={
        \sseq@keys@lastpart
        \edef\sseq@temppath{\pgfkeyscurrentpath/\sseq@temp}
        \pgfkeysalso{\sseq@temppath/.code/.expanded={\@nx\sseq@error@n{option-illegal}{\sseq@temp}}}
    },
    .sseq @ illegal as global/.code={
        \sseq@keys@lastpart
        \pgfkeysalso{/sseqpages/global/\sseq@temp/.code/.expanded={\@nx\sseq@error@n{option-illegal-as-global}{\sseq@temp}}}
    },
    .sseq @ global only/.code={
        \sseq@keys@lastpart
        \pgfkeysalso{/sseqpages/\sseq@temp/.code/.expanded={\@nx\sseq@error@n{option-global-only}{\sseq@temp}}}
    }
}


%%% Some easy internal macros

% This is for .unknown handlers. Try same key in different directory #1.
\def\sseq@passoptionto#1{\sseq@eval{\@nx\pgfqkeys{#1}{\pgfkeyscurrentname={\unexpanded\@xp{\pgfkeyscurrentvalue}}}}}

% Immediately execute a key that is either a tikz or pgf key.
% #1 -- key
% #2 -- value
\def\sseq@pgfkeysdocommand#1#2{
    \sseq@tempiffalse
    \sseq@pgfkeysdocommand@try{/tikz}{#1}{#2}
    \sseq@pgfkeysdocommand@try{/pgf}{#1}{#2}
    \ifsseq@tempif\else\sseq@error@internal@nn{do-command-failed}{#1}{#2}\fi
}

% Execute a key if \sseq@tempif is true and it is defined, if so set \sseq@tempif to false.
% These are meant to cascade.
% #1 -- directory
% #2 -- key
% #3 -- value
\def\sseq@pgfkeysdocommand@try#1#2#3{
    \ifsseq@tempif\else
        \pgfkeysifdefined{#1/#2/.@cmd}{
            \sseq@tempiftrue
            \def\pgfkeysdefaultpath{#1/}
            \edef\pgfkeyscurrentkey{#1/#2}
            \pgfkeysgetvalue{#1/#2/.@cmd}{\pgfkeys@code}%
            \pgfkeys@code#3\pgfeov
        }{
            \pgfkeysifdefined{#1/#2}{
                \sseq@tempiftrue
                \pgfkeyssetvalue{#1/#2}{#3}
            }{}
        }
    \fi
}

%%%
% Our main unknown option handler
% This is copied from the standard tikz .unknown handler, but instead of running the code it stores it in \sseq@savedoptioncode.
% When it tries the key in /tikz and in /pgf it also recursively expands \pgfkeysalso's.

\def\sseq@keys@unknown@error@normal{\sseq@error{unknown-key}} % a hook so we can turn off errors if we like
\let\sseq@keys@unknown@error\sseq@keys@unknown@error@normal % This is used in .sseq style
\let\sseq@add@tikzkey@to@options\sseq@e@addto@options % another hook for .sseq style

\def\sseq@keys@unknown{%
    \let\tikz@key\pgfkeyscurrentname
    \sseq@keys@try@addtooptions{/sseqpages/userstyles,/tikz,/pgf}{}{%
        \@xp\pgfutil@in@\@xp!\@xp{\tikz@key}%
        \ifpgfutil@in@%
          % this is a color!
            \sseq@keys@unknown@handlecolor
        \else%
          \pgfutil@doifcolorelse{\tikz@key}%
          {%
            \sseq@keys@unknown@handlecolor
          }%
          {%
            % Ok, second chance: This might be an arrow specification:
            \@xp\pgfutil@in@\@xp-\@xp{\tikz@key}%
            \ifpgfutil@in@%
              % Ah, an arrow spec!
              \sseq@add@tikzkey@to@options{\@nx\sseq@options@secondpass{\@nx\tikz@processarrows{\tikz@key}}}%
            \else%
              % Ok, third chance: A shape!
              \@xp\ifx\csname pgf@sh@s@\tikz@key\endcsname\relax%
                    \sseq@keys@unknown@error%
              \else%
                \sseq@add@tikzkey@to@options{\@nx\sseq@options@secondpass{\def\@nx\tikz@shape{\tikz@key}}}%
              \fi%
            \fi%
          }%
      \fi%
    }%
}

\def\sseq@keys@unknown@handlecolor{%
  \sseq@add@tikzkey@to@options{\@nx\sseq@options@secondpass{\@nx\tikz@addoption{\@nx\tikz@compat@color@set{\tikz@key}}}}%
  \sseq@add@tikzkey@to@options{\@nx\sseq@options@secondpass{\def\@nx\tikz@textcolor{\tikz@key}}}%
}

%%% \sseq@keys@try@addtooptions
% #1 -- a comma separated list of directories
% #2 -- success code
% #3 -- failure code
%
% This is for the main .unknown handler and for \sseq@addstylecodetooptions (which pretends to be \pgfkeys executed in a directory with only a .unknown)
% Tries to run the current key (\pgfkeyscurrentname=\pgfkeyscurrentvalue) in each of the directories in turn.
% If it finds the key in one of the directories, recursively expand the \pgfkeysalso in the code,
% add the final result to \sseq@savedoptioncode, and run success code. Otherwise run failure code.
%
% The goal is to avoid storing any commands that require the use of pgfkeys.
% This is only a partial solution, because it misses \tikzset, \pfgqkeys, etc.
% \pgfkeysalso is particularly dangerous though because we might not remember what directory it was supposed to happen in.
% In order to get the recursive expansion of \pgfkeysalso's we need to modify the pgfkeys algorithm so that it
% passes control back to \sseq@keys@addtooptions@checkalso when it's done.
% Modifies tikz commands \pgfkeys@case@one and \pgfkeys@case@two@extern from file /pgf/utilities/pgfkeys.code.tex lines 352 and 365
\def\sseq@keys@try@addtooptions#1{%
    \sseq@keys@try@addtooptions@#1,\sseq@nil
}
\def\sseq@keys@try@addtooptions@#1,{%
    \bgroup
    \def\pgfkeysdefaultpath{#1/}%
    \def\next{\egroup\@ifnextchar\sseq@nil{\@xp\@secondoftwo\@gobble}{\sseq@keys@try@addtooptions@}}%
    \pgfkeysifdefined{#1/\pgfkeyscurrentname/.@cmd}{%
        \edef\pgfkeyscurrentkey{#1/\pgfkeyscurrentname}
        \ifx\pgfkeyscurrentvalue\pgfkeysnovalue@text% Hmm... no value
          \pgfkeysifdefined{#1/\pgfkeyscurrentname/.@def}%
          {\pgfkeysgetvalue{#1/\pgfkeyscurrentname/.@def}{\pgfkeyscurrentvalue}}
          {}% no default, so leave it
        \fi%
        \pgfkeysgetvalue{#1/\pgfkeyscurrentname/.@cmd}{\pgfkeys@code}%
        \sseq@eval{\@nx\sseq@keys@addtooptions@checkalso{\unexpanded\@xptwo{\@xp\pgfkeys@code\pgfkeyscurrentvalue\pgfeov}}}%
        \sseq@smuggle@macro\sseq@savedoptioncode
        \egroup
        \def\next{\@xp\@firstoftwo\sseq@gobble@to@nil}%
    }{%
        \pgfkeysifdefined{#1/\pgfkeyscurrentname}{%
            \egroup
            \sseq@add@tikzkey@to@options{\@nx\sseq@options@secondpass{\@nx\pgfkeyssetvalue{#1/\pgfkeyscurrentname}{\unexpanded\@xp{\pgfkeyscurrentvalue}}}}%
            \def\next{\@xp\@firstoftwo\sseq@gobble@to@nil}%
        }{}%
    }%
    \next
}

% Iterate over list and run \pgfkeysalso's when we find them. Note that we use our variants of \pgfkeys@case@... which hand control back to this command,
% so this is recursive -- the result is guaranteed not to have any instance of \pgfkeysalso in it!
\def\sseq@keys@addtooptions@checkalso#1{%
    \let\pgfkeys@case@one\sseq@pgfkeys@case@one@store
    \let\pgfkeys@case@two@extern\sseq@pgfkeys@case@two@extern@store
    \sseq@keys@addtooptions@checkalso@#1\pgfkeysalso\sseq@nil
}

\def\sseq@keys@addtooptions@checkalso@#1\pgfkeysalso#2{%
    \def\sseq@tempa{/tikz/}
    \def\sseq@tempb{/pgf/}
    % This is an extra hook I needed to add so that my own keys can execute code when they need to...
    \sseq@keys@addtooptions@checkdo#1\sseq@keys@do\sseq@nil
    \ifx\sseq@nil#2% We're done, restore pgfkeys.
        \let\pgfkeys@case@one\sseq@pgfkeys@case@one@save
        \let\pgfkeys@case@two@extern\sseq@pgfkeys@case@two@extern@save
    \else
        \ifx\expandafter#2 % if they wanted to expand the first token, handle that
            \sseq@add@tikzkey@to@options{\sseq@options@secondpass@iftikz{\relax\relax}} % The last token was an \expandafter, we need to keep it from doing anything
            \@xptwo\sseq@keys@addtooptions@checkalso@xp
        \else
            \pgfkeysalso{#2}%
            \@xptwo\sseq@keys@addtooptions@checkalso@
        \fi
    \fi
}

\def\sseq@keys@addtooptions@checkalso@xp#1{
    \@xp\pgfkeysalso\@xp{#1}
    \sseq@keys@addtooptions@checkalso@
}

% Used to get collections code to run right.
\def\sseq@keys@addtooptions@checkdo#1\sseq@keys@do#2{
    \sseq@add@tikzkey@to@options{\sseq@options@secondpass@iftikz{\unexpanded{#1}}}
    \ifx#2\sseq@nil\else
        #2%
        \@xp\sseq@keys@addtooptions@checkdo
    \fi
}

% This has to be expandable, so \sseq@tempa and \sseq@tempb are set to "/tikz/" and "/pgf/" ahead of time
% Defer the command if it is in directories /pgf or /tikz, otherwise run it now. TODO: explain here why we need this.
\def\sseq@options@secondpass@iftikz{%
    \ifx\pgfkeysdefaultpath\sseq@tempa
        \@nx\sseq@options@secondpass
    \else
        \ifx\pgfkeysdefaultpath\sseq@tempb
            \@nx\sseq@options@secondpass
        \else
            \@xptwo\@firstofone
        \fi
    \fi
}


% Here we hook into pgfkeys. This is remarkably easy!
\let\sseq@pgfkeys@case@one@save\pgfkeys@case@one
\let\sseq@pgfkeys@case@two@extern@save\pgfkeys@case@two@extern
\def\sseq@pgfkeys@case@one@store{%
    \pgfkeysifdefined{\pgfkeyscurrentkey/.@cmd}{%
        \pgfkeysgetvalue{\pgfkeyscurrentkey/.@cmd}{\pgfkeys@code}%
        % The following line is the only change, it used to be \@xp\pgfkeys@code\pgfkeyscurrentvalue\pgfeov which would just run the code.
        % Instead we recursively expand the \pgfkeysalso's and then store in it \sseq@savedoptioncode
        \sseq@eval{\@nx\sseq@keys@addtooptions@checkalso{\unexpanded\@xptwo{\@xp\pgfkeys@code\pgfkeyscurrentvalue\pgfeov}}}%
    }{\pgfkeys@case@two}%
}
\def\sseq@pgfkeys@case@two@extern@store{%
  \ifx\pgfkeyscurrentvalue\pgfkeysnovalue@text%
    \pgfkeysvalueof{\pgfkeyscurrentkey}% this probably isn't a real case
  \else%
    % The following line is the only change, it used to be (roughly) \pgfkeyssetvalue{\pgfkeyscurrentkey}{\pgfkeyscurrentvalue}
    % which would immediately set the value. Instead we store in it \sseq@savedoptioncode
    \sseq@add@tikzkey@to@options{\@nx\sseq@options@secondpass{\@nx\pgfkeyssetvalue{\pgfkeyscurrentkey}{\unexpanded\@xp{\pgfkeyscurrentvalue}}}}%
  \fi%
}

\pgfkeys{%
    /sseqpages/.is family,
    /sseqpages/.unknown/.code={\sseq@keys@unknown},
%
    /sseqpages/class/.unknown/.code={%
        \expandafter\ifx\csname pgf@sh@s@\pgfkeyscurrentname\endcsname\relax % is it the name of a shape?
            \sseq@passoptionto{/sseqpages}%
        \else
            \sseq@e@addto@options{\def\@nx\tikz@shape{\pgfkeyscurrentname}}% if so, store it
        \fi
    },
    /sseqpages/.is family,
    /sseqpages/scope/.is family,
    /sseqpages/scope/.unknown/.code={\sseq@passoptionto{/sseqpages}},
    /sseqpages/class/.is family,
    /sseqpages/class/label/.is family,
    /sseqpages/class/label/.unknown/.code={\sseq@keys@unknown},
    /sseqpages/class/label/pin/.unknown/.code={\sseq@keys@unknown},
    /sseqpages/class/node texts/.is family,
    /sseqpages/class/node texts/.unknown/.code={\sseq@passoptionto{/tikz}},
    /sseqpages/differential/.is family,
    /sseqpages/differential/.unknown/.code={\sseq@passoptionto{/sseqpages}},
    /sseqpages/struct line/.is family,
    /sseqpages/struct line/.unknown/.code={\sseq@passoptionto{/sseqpages}},
    /sseqpages/label/.is family,
    /sseqpages/label/.unknown/.code={\sseq@keys@unknown},
    /sseqpages/tikz primitives/.is family,
    /sseqpages/tikz primitives/.unknown/.code={\sseq@keys@unknown},
    % One off directories
    /sseqpages/fit/.is family,
    /sseqpages/fit/.unknown/.code={\sseq@passoptionto{/sseqpages}},
    /sseqpages/foreach/.unknown/.code={\sseq@passoptionto{/pgf/foreach}},
    /sseqpages/class placement/.unknown/.code={\sseq@passoptionto{/tikz}}
}

\pgfqkeys{/sseqpages/class}{shape/.code={\edef\tikz@shape{\pgfkeyscurrentname}}} % shape=whatever shape passing

\newif\ifsseq@pin
\pgfqkeys{/sseqpages/class/label}{% maybe implement multipart node labels this way? No way!
    node/.sseq @ first pass code={\sseq@classlabelfalse},
    inside/.sseq @ first pass code={\sseq@classlabelfalse},
    pin/.sseq @ first pass code = {
        \let\sseq@savedoptioncode@save\sseq@savedoptioncode
        \sseq@processoptions{class/label/pin}{#1}
        \edef\sseq@savedoptioncode{\unexpanded\@xp{\sseq@savedoptioncode@save}\@nx\sseq@d@addto@macro\@nx\sseq@pinoptions{\unexpanded\@xp{\sseq@savedoptioncode}}}
    },
    pin/.sseq @ second pass append code = {\sseq@pintrue}
}
\def\sseq@pinoptions{}

%% label directions
\def\sseq@labeldirectionlist{\\{above}\\{below}\\{left}\\{right}\\{above left}\\{above right}\\{below left}\\{below right}}
\def\sseq@setlabeldirectionkeys#1{
    \@ifnextchar[{\sseq@setlabeldirectionkeys@{#1}}{\sseq@setlabeldirectionkeys@{#1}[/.code]}%]
}
\def\sseq@setlabeldirectionkeys@#1[#2]#3{
    \def\sseq@temp##1{#3}
    \sseq@temptoks{}
    \def\\##1{
        \sseq@eval{\sseq@temptoks{\the\sseq@temptoks##1#2={\unexpanded\@xp{\sseq@temp{##1}}},}}
    }
    \sseq@labeldirectionlist
    \sseq@eval{\@nx\pgfqkeys{#1}{\the\sseq@temptoks}}
}

\def\\#1{\@xp\sseq@addtostorelist\csname sseq@default@#1@distance\endcsname{1ex}}
\sseq@labeldirectionlist
\def\sseq@defaultlabeldistance#1{\@xp\csname sseq@default@#1@distance\endcsname}
\def\sseq@setdefaultlabeldistance#1#2{\sseq@savedpaths@add{\@xp\gdef\csname sseq@default@#1@distance\endcsname{#2}}}
\def\sseq@setdefaultlabeldistances#1{\def\\##1{\sseq@setdefaultlabeldistance{##1}{#1}}\sseq@labeldirectionlist}



\sseq@setlabeldirectionkeys{/sseqpages/class/label}[/.sseq @ first pass code]{\sseq@classlabeltrue}
\sseq@setlabeldirectionkeys{/sseqpages/class/label}[/.sseq @ second pass append code]{\sseq@class@positionlabel{#1}{##1}}
\sseq@setlabeldirectionkeys{/sseqpages/class/label}[/.default]{\sseq@defaultlabeldistance{#1}}
\sseq@setlabeldirectionkeys{/sseqpages/global/class/label}[/.default]{\sseq@defaultlabeldistance{#1}}

\sseq@setlabeldirectionkeys{/sseqpages/global}[ label distance/.code]{\sseq@setdefaultlabeldistance{#1}{##1}}
\sseq@setlabeldirectionkeys{/sseqpages}[ label distance/.code]{\sseq@setdefaultlabeldistance{#1}{##1}}

\sseq@pgfqkeysdirlist{/sseqpages/global,/sseqpages}{
    label distance/.code={\sseq@setdefaultlabeldistances{#1}}
}

\def\sseq@class@positionlabel#1#2{\ifx\tikz@fig@name\undefined\else\sseq@pgfkeysdocommand{#1}{#2 of \tikz@fig@name}\fi}



\sseq@addtostorelist\sseq@classlabeltextfn{} % \sseq@classlabeltextfn is public via "class label handler"
\def\sseq@classlabeltextfn@default#1{\def\processedlabel{#1}}
\def\sseq@labeltextfn#1{#1} % labeltextfn is private
\def\sseq@labeltextfn@rotatescale#1{%
    \@nx\ifmmode
        \@nx\scalebox{\sseq@labletext@scale@factor}{\@nx\rotatebox{\sseq@labletext@rotate@angle}{$#1$}}%
    \@nx\else
        \@nx\scalebox{\sseq@labletext@scale@factor}{\@nx\rotatebox{\sseq@labletext@rotate@angle}{#1}}%
    \@nx\fi
}
\def\sseq@labletext@rotate@angle{0}
\def\sseq@labletext@scale@factor{1}

\bgroup\lccode`8=`p\lccode`9=`t\lowercase{\egroup
\def\sseq@eatpt#189{#1}
}

\sseq@pgfqkeysdirlist{/sseqpages/label,/sseqpages/class/label}{
    rotate/.code={%
        \let\sseq@labeltextfn\sseq@labeltextfn@rotatescale
        \edef\sseq@labletext@rotate@angle{\@xp\sseq@eatpt\the\dimexpr\sseq@labletext@rotate@angle pt+#1pt}%
    },
    scale/.code={%
        \let\sseq@labeltextfn\sseq@labeltextfn@rotatescale
        \edef\sseq@labletext@scale@factor{\@xp\sseq@eatpt\the\dimexpr\sseq@labletext@scale@factor pt*#1}%
    },
%    shift/.sseq @ second pass code={\pgfkeysalso{xshift=\sseq@shift@getx#1}\pgfkeysalso{yshift=\sseq@shift@gety#1}}
}


\pgfqkeys{/sseqpages/global}{
    background color/.initial=white
}
\pgfqkeys{/sseqpages/label}{
    description/.code={%
        \edef\sseq@edgeoptionhook{\@nx\pgfkeys{/tikz/fill=\pgfkeysvalueof{/sseqpages/global/background color}}}%
    },
    description/.sseq @ second pass append code={%
        \pgfkeysalso{/tikz/anchor=center}%
    }
}

\def\sseq@edgeoptionhook{}% This is just to make description work.

%%  Labels
%%  The code for this that actually does most of the work is inside of sseqmain.code.tex -- search for "Labels"
%%  This just handles the interface with the keyvalue system here.


%% Quote handling
% \sseq@handlequote will be installed later as the 'first char syntax' for ".
% Problem is that 'first char syntax' isn't allowed to care what directory it is in, but we need different handling
% for classes versus edges versus scopes. Solution is to use \pgfkeysalso and then define the handler in a key.
\def\sseq@protectlabel{\@nx\sseq@protectlabel\@nx}
\def\sseq@handlequote#1{\pgfkeysalso{handle quote=#1}}
\pgfkeys{%
    /sseqpages/class/handle quote/.code={%
        \let\tikz@quotes@as@save\tikz@quotes@as
        \let\tikz@quotes@as\sseq@handleclassquotes
        \tikz@quote@parser{#1}%
        \let\tikz@quotes@as\tikz@quotes@as@save
    },
    /sseqpages/handle quote/.code={%
        \let\tikz@quotes@as\sseq@handleedgequotes
        \tikz@quote@parser{#1}%
        \let\tikz@quotes@as\tikz@quotes@as@save
    }
}


\def\sseq@handlescopequotes#1#2{% We use this via \let\tikz@quotes@as\sseq@handleclassquotes
    /utils/exec={% so it's called inside \pgfkeysalso{stuff}. Need to surround with /utils/exec={} to get out.
       \global\let\sseq@classnodetext\pgfutil@empty%
       \global\let\sseq@classnodetextoptions\pgfutil@empty%
       \global\let\sseq@classlabelnodes\pgfutil@empty%
       \sseq@handleclassquotes@inner{#1}{#2}%
    }
}

% We use this via \let\tikz@quotes@as\sseq@handleclassquotes
% so it's called inside \pgfkeysalso{stuff}. Need to surround with /utils/exec={} to get out.
\def\sseq@handleclassquotes#1#2{/utils/exec={\sseq@d@addto@macro\sseq@processlabels{\sseq@handleclassquotes@inner{#1}{#2}}}}



\def\sseq@checktransform@error{\sseq@error@nn{illegal-transform}}
\let\sseq@checktransform\sseq@checktransform@error


%% Parsing, background vs foreground, and context
\def\sseq@standardparse{%
    \let\sseq@tikzprimitives@coords@maybeclass\sseq@tikzprimitives@coords@notaclass
    \let\sseq@checktransform\@gobbletwo
    \let\sseq@tikz@shifttransform\sseq@okayshifttransform
    % TODO: This needs fixing:
    %((((((((((((((((((
    \def\class##1){\sseq@setthiscall{\string\class##1)}\sseq@error@x{standard-parse}{\string\class}}%
    \def\replaceclass##1){\sseq@setthiscall{\string\replaceclass##1)}\sseq@error@x{standard-parse}{\string\replaceclass}}%
    \def\classoptions##1){\sseq@setthiscall{\string\classoptions##1)}\sseq@error@x{standard-parse}{\string\classoptions}}%
    \def\d##1){\sseq@setthiscall{\string\d##1)}\sseq@error@x{standard-parse}{\string\d}}%
    \def\doptions##1){\sseq@setthiscall{\string\doptions##1)}\sseq@error@x{standard-parse}{\string\doptions}}%
    \def\structline##1)##2){\sseq@setthiscall{\string\structline##1)##2)}\sseq@error@x{standard-parse}{\string\structline}}%
    \def\structlineoptions##1)##2){\sseq@setthiscall{\string\structlineoptions##1)##2)}\sseq@error@x{standard-parse}{\string\structlineoptions}}%
}

% I don't think this is ever actually used...
\def\sseq@nodeparse{%
    \let\sseq@tikzprimitives@coords@maybeclass\sseq@tikzprimitives@coords@maybeclass@save
    \let\sseq@tikzprimitives@coords@maybeclass
    \def\sseq@transform{\sseq@error@nn{illegal-transform}} % Disallow most coordinate transforms
    \let\sseq@shifttransform\sseq@checkshifttransform
    \let\class\sseq@class
    \let\replaceclass\sseq@replaceclass
    \let\structline\sseq@structline
}


\sseq@pgfqkeysdirlist{/sseqpages,/sseqpages/tikz primitives}{%
    standard parse/.code={
        \let\sseq@firstpass@shifttransform\@gobbletwo
        \sseq@d@addto@options{\sseq@standardparse}
        \pgfkeys{/sseqpages/tikz primitives/shift/.sseq @ second pass code={\tikzset{shift={##1}}}}
    },
    node parse/.code={ % This is never used, probably broken, and likely shouldn't exist
        \let\sseq@firstpass@shifttransform\sseq@firstpass@checkshifttransform
        \let\sseq@checktransform\sseq@checktransform@error
        \sseq@d@addto@options{\sseq@nodeparse}
    },
    background/.code={\sseq@setsavedpaths{background}\pgfkeysalso{standard parse}},
    foreground/.code={\ifsseq@keepchanges\sseq@setsavedpaths{standard}\else\sseq@setsavedpaths{temporary}\fi\pgfkeysalso{node parse}}
}

\pgfkeys{
    /sseqpages/tikz primitives/handle quote/.sseq @ second pass code={\tikzset{#1}},
    /tikz/handle quote/.code={\sseq@error{tikz-edge-quotes}}
}

\sseq@pgfqkeysdirlist{/sseqpages/class,/sseqpages/differential}{
    standard parse/.sseq @ first pass code={\sseq@error@n{option-tikz-prims-only}{standard parse}},
    background/.sseq @ first pass code={\sseq@error@n{option-tikz-prims-only}{background}}
}

% Some class options. All of these will make changes local to the current \class command. They are communicated to \class via these commands,
% which we initialize here as empty.
\def\sseq@class@tag{}
\def\sseq@class@tagprefix{}
\def\sseq@class@name{}
\def\sseq@namehandler{}
\def\sseq@currenttag{\sseq@class@tag}% for the user ??
\def\sseq@tooltip{}
\pgfqkeys{/sseqpages/scope}{
    tag/.code={\pgfkeys@spdef\sseq@class@tag{#1}},
    tag prefix/.code={\pgfkeys@spdef\temp{#1}\sseq@e@addto@macro\sseq@class@tagprefix{\unexpanded\@xp{#1}}}
}

\def\sseq@classinsert{}
\pgfqkeys{/sseqpages/class}{
    name/.code={\def\sseq@class@name{#1}\pgfkeysalso{/sseqpages/alias=#1}},
    show name/.store in=\sseq@class@showname,
    page/.code={\sseq@getnumrange\sseq@class@page{#1}},% page & gen are only for \classoptions.
    generation/.code={\sseq@getnumrange\sseq@gen{#1}}, % \class tests if \sseq@class@page is set and throw an error if so.
    tag/.code={\sseq@d@addto@macro\sseq@class@tag{#1}},
    offset/.store in=\sseq@tempoffset,
    tooltip/.sseq @ first pass code={\def\sseq@tooltip{#1}},
    insert/.store in=\sseq@classinsert
}

\pgfqkeys{/sseqpages/struct line}{
    page/.code={\sseq@getnumrange\sseq@structline@page{#1}}
}

\def\sseq@fitalso{}
\pgfqkeys{/sseqpages/fit}{
    fit/.sseq @ first pass code={\sseq@d@addto@macro\sseq@fitalso{#1}},
    page/.code={\sseq@getnumrange\sseq@fit@page{#1}}
}

\def\sseq@getnumrange#1#2{
    \pgfutil@in@{--}{#2}\ifpgfutil@in@
        \sseq@getnumrange@range#1#2\sseq@nil
    \else
        \edef#1{\the\numexpr#2}
    \fi
}
\def\sseq@getnumrange@range#1#2--#3\sseq@nil{\edef#1{\the\numexpr#2}\@xp\edef\csname\sseq@macroname#1max\endcsname{\the\numexpr#3}}

%% Needs tikz
% Force \class or \d to fall back to using tikz to render their output
% Rather than using the faster direct \pgf primitives (largely copied from tikz to maximize compatibility)
% Drawing with tikz is slower, but I don't want to reimplement these features
\def\sseq@needstikzandpass#1#2{\sseq@needstikzandpass@{#1}#2,\@nil}
\def\sseq@needstikzandpass@#1#2,{%
    \pgfqkeys{#1}{#2/.code={\sseq@options@firstpass{\sseq@needstikztrue}\sseq@options@secondpass{\sseq@pgfkeysdocommand{#2}{##1}\sseq@needstikztrue}}}%
    \@ifnextchar\@nil{\@gobble}{\sseq@needstikzandpass@{#1}}%
}

\pgfqkeys{/sseqpages}{needs tikz/.code=\sseq@needstikztrue}
% All of these things need tikz to draw but otherwise require no special handling.
\sseq@needstikzandpass{/sseqpages}{%
    shade, blend mode, pattern, path picture, path fading, decorate, pin, label % should pin be here?
}

\sseq@needstikzandpass{/sseqpages/differential}{%
    bend left, bend right, in, out, relative, looseness, in looseness, out looseness,
    min distance, max distance, out min distance, out max distance, in min distance, in max distance, distance,
    controls, in control, out control,
    loop, loop above, loop below, loop left, loop right % maybe delete the loops and replace them with errors? It's pretty pointless to use these.
}

\sseq@needstikzandpass{/sseqpages/struct line}{%
    bend left, bend right, in, out, relative, looseness, in looseness, out looseness,
    min distance, max distance, out min distance, out max distance, in min distance, in max distance, distance,
    controls, in control, out control,
    loop, loop above, loop below, loop left, loop right % maybe delete the loops and replace them with errors? It's pretty pointless to use these.
}

%    circle split/.code=\sseq@needstikzandpass,
%    circle solidus/.code=\sseq@needstikzandpass,
%    ellipse split/.code=\sseq@needstikzandpass,
%    rectangle split/.code=\sseq@needstikzandpass,
%    align/.code=\sseq@needstikzandpass,

\sseq@addtostorelist\sseq@runoffarrow@start@differential@spec{}
\sseq@addtostorelist\sseq@runoffarrow@end@differential@spec{>}
\sseq@addtostorelist\sseq@runoffarrow@start@structline@spec{...}
\sseq@addtostorelist\sseq@runoffarrow@end@structline@spec{...}
\sseq@pgfqkeysdirlist{/sseqpages/global,/sseqpages}{
    run off differentials/.code args={#1-#2}{\def\sseq@runoffarrow@start@differential@spec{#1}\def\sseq@runoffarrow@end@differential@spec{#2}},
    run off struct lines/.code args={#1-#2}{\def\sseq@runoffarrow@start@structline@spec{#1}\def\sseq@runoffarrow@end@structline@spec{#2}},
    run off/.forward to=/sseqpages/run off differentials,
    run off/.forward to=/sseqpages/run off struct lines
}


\sseq@pgfqkeysdirlist{/sseqpages/differential,/sseqpages/struct line}{
    source anchor/.sseq @ second pass store in=\sseq@edgesourceanchor,
    target anchor/.sseq @ second pass store in=\sseq@edgetargetanchor,
    shorten >/.sseq @ second pass code={\sseq@pgfkeysdocommand{shorten >}{#1}},
    shorten </.sseq @ second pass code={\sseq@pgfkeysdocommand{shorten <}{#1}},
}
\pgfqkeys{/sseqpages/differential}{
    /sseqpages/differential/invisible/.sseq @ first pass code={\sseq@drawdifferentialfalse}
}
\newif\ifsseq@drawdifferential
\sseq@drawdifferentialtrue

%%%
%%%   Global Keys
%%%

%%% Style options

%% Add key code to \sseq@savedoptioncode
% #1 -- a list of directories to search
% #2 -- keys
%
% This is pretty similar in basic concept to \sseq@processoptions, but instead of using the .unknown handlers to walk up the directory,
% it just tries the key in each directory in the list. I'm not actually sure anymore why it was important to do it this way, but there probably was a good reason...
%
% This is a faster and sloppier way of doing:
% \def\sseq@temp{#1}
% \pgfqkeys{/sseqpages/alwaysunknown}{#2}
% and defining an event handler: alwaysunknown/.unknown/.code={\@xp\sseq@keys@try@addtooptions\@xp{\sseq@temp}{}{\sseq@keys@unknown}
% That basically takes advantage of the pgfkeys parser but gives me back control over what to do with the parsed things.
%
% For some reason I made the probably misguided decision that that approach was too slow, even though this isn't performance
% critical code. Maybe I had a better reason for doing this that I can't remember. Instead I copied pgfkeys but deleted all the
% parts that are useless in this context, which is most of it. Occasionally bugs crop up here when it turns out I deleted a little too much.
%
% This uses a hook \sseq@addstylecodetooptions@unknown to allow the sseqs style to sift out the nonglobal keys and handle them differently.
%
% Copied with modification from /pgf/utilities/pfgkeys.code.tex

\def\sseq@addstylecodetooptions#1#2{%
    \def\sseq@directorylist{#1}%
    \sseq@simple@pgfkeys@parse#2,\pgfkeys@mainstop
}

\def\sseq@simple@pgfkeys@parse{\futurelet\pgfkeys@possiblerelax\sseq@simple@pgfkeys@parse@main}
\def\sseq@simple@pgfkeys@parse@main{%
  \ifx\pgfkeys@possiblerelax\pgfkeys@mainstop%
      \@xp\@gobble
  \else%
      \@xp\sseq@simple@pgfkeys@@normal
  \fi%
}

\def\sseq@simple@pgfkeys@@normal#1,{%
    \sseq@simple@pgfkeys@unpack#1=\pgfkeysnovalue=\pgfkeys@stop
    \sseq@simple@pgfkeys@parse
}

\def\sseq@simple@pgfkeys@unpack#1=#2=#3\pgfkeys@stop{%
  \pgfkeys@spdef\pgfkeyscurrentkey{#1}%
  \pgfkeys@spdef\pgfkeyscurrentvalue{#2}%
  \ifx\pgfkeyscurrentkey\pgfkeys@empty\else
      \def\next{\@xp\sseq@keys@try@addtooptions\@xp{\sseq@directorylist}{}{\sseq@addstylecodetooptions@unknown}}%
      % We deleted the general "first char syntax" handler, so we need to do our own check for the quote.
      \sseq@simple@pgfkeys@checkquote#1\sseq@nil{%
            \def\pgfkeyscurrentname{handle quote}%
            \def\pgfkeyscurrentvalue{#1}%
      }{%
          \edef\pgfkeyscurrentkey{\pgfkeyscurrentkey}%
          \ifx\pgfkeyscurrentkey\pgfkeys@empty
            \let\next\relax
          \else%
            \let\pgfkeyscurrentname\pgfkeyscurrentkey
          \fi
      }%
      \@xp\next
  \fi
}
% This is a hook for the "sseqs" style.
\let\sseq@addstylecodetooptions@unknown\sseq@keys@unknown

\long\def\sseq@simple@pgfkeys@checkquote#1#2\sseq@nil{%
    \ifx#1"%
        \@xp\@firstoftwo
    \else
        \@xp\@secondoftwo
    \fi
}


\def\sseq@stylelist{}
\def\sseq@clearchangestyles{}

% The basic idea here is we make a tokens list \sseq@structlinestyle and define a key "struct lines" that takes a list of keys
% and adds the keys to \sseq@structlinestyle. In order to do this, I need some special trick to delete spaces.
% The macro \sseq@setmacronospaces defined in spectralsequences.sty accomplishes this.

\def\sseq@newstyles#1#2{\def\sseq@temppath{#1}\sseq@newstyles@#2,\sseq@nil}
\def\sseq@newstyles@#1,{\sseq@newsseqstyle{\sseq@trimleadingspaces#1}\@ifnextchar\sseq@nil{\@gobble}{\sseq@newstyles@}}

\def\sseq@set@newstyle#1{\pgfqkeys{/sseqpages}{#1}\pgfqkeys{/sseqpages/global}{#1}\pgfqkeys{/sseqpages/global/default}{#1}}

% Here are two useful wrappers around addstylecodetotoks.

% #1 -- token list
% #2 -- path
% #3 -- code
\def\sseq@addstylecodetotoks#1#2#3{
    \let\sseq@savedoptioncode@store\sseq@savedoptioncode
    \edef\sseq@savedoptioncode{\the#1}%
    \sseq@addstylecodetooptions{#2}{#3}%
    #1\@xp{\sseq@savedoptioncode}%
    \let\sseq@savedoptioncode\sseq@savedoptioncode@store
}

% #1 -- macro
% #2 -- path
% #3 -- code
\def\sseq@addstylecodetomacro#1#2#3{
    \let\sseq@savedoptioncode@store\sseq@savedoptioncode
    \let\sseq@savedoptioncode#1%
    \sseq@addstylecodetooptions{#2}{#3}%
    \let#1\sseq@savedoptioncode
    \let\sseq@savedoptioncode\sseq@savedoptioncode@store
}

\def\sseq@newsseqstyle#1{%
    \sseq@setmacronospaces\sseq@tempcmd{sseq@#1style}%
    \edef\sseq@tempcmdb{\csname\@xp\sseq@macroname\sseq@tempcmd @page\endcsname}%
    \sseq@setmacronospaces\sseq@tempcmdc{sseq@the#1style}%
    \@xp\edef\sseq@tempcmdc{\@nx\the\@xp\@nx\sseq@tempcmd\@nx\the\sseq@tempcmdb}%
    \@xp\newtoks\sseq@tempcmd
    \@xp\newtoks\sseq@tempcmdb
    \sseq@x@addto@macro\sseq@stylelist{\sseq@tempcmd{\@nx\the\sseq@tempcmd}}%
    \sseq@x@addto@macro\sseq@clearchangestyles{\sseq@tempcmdb{}}%
    \sseq@set@newstyle{%
        #1s/.ecode={%
            \@nx\sseq@addstylecodetotoks{\sseq@tempcmd}{\sseq@temppath}{##1}
            \@nx\ifsseq@inprogress
                \@nx\sseq@savedpaths@xadd{\sseq@tempcmd{\@nx\the\sseq@tempcmd}}%
            \@nx\fi
        },%
        change #1s/.ecode={%
            \@nx\ifsseq@ispageenv\@nx\else\@nx\sseq@error@nn{change-style-data}{#1}{##1}\@nx\fi
            \@nx\ifsseq@hasname\@nx\else\@nx\sseq@error@nn{change-style-standalone}{#1}{##1}\@nx\fi
            \@nx\sseq@addstylecodetotoks{\sseq@tempcmdb}{\sseq@temppath}{##1}
        }
    }%
}


\newtoks\sseq@sseqstyle
\newtoks\sseq@sseqstyle@page
\def\sseq@thesseqstyle{\the\sseq@sseqstyle\the\sseq@sseqstyle@page}

% sseqs has to be handled separately, because it must decide whether an option is a global option or not
% and handle them at different times. This makes the code a little trickier.
\def\sseq@sseqstyle@globalpart{}
\sseq@g@addto@macro\sseq@stylelist{\sseq@sseqstyle{\the\sseq@sseqstyle}}%
\sseq@g@addto@macro\sseq@clearchangestyles{\sseq@sseqstyle@page{}}%
\sseq@set@newstyle{
    sseqs/.code={%
        % This will get called on each key that isn't a defined global option. Then we add it to \sseq@sseqstyle to be applied
        % locally to absolutely everything.
        \def\sseq@addstylecodetooptions@unknown{%
            \let\sseq@sseqstyle@globalpart\sseq@savedoptioncode
            \edef\sseq@savedoptioncode{\the\sseq@sseqstyle}%
            \sseq@keys@unknown
            \sseq@sseqstyle\@xp{\sseq@savedoptioncode}%
            \let\sseq@savedoptioncode\sseq@sseqstyle@globalpart
        }%
        \sseq@addstylecodetomacro\sseq@sseqstyle@globalpart{/sseqpages/global}{#1}%
        \ifsseq@inprogress
            \sseq@savedpaths@xadd{\sseq@sseqstyle{\the\sseq@sseqstyle}}%
        \fi
        % Restore hook
        \let\sseq@addstylecodetooptions@unknown\sseq@keys@unknown
    },%
    % It's not clear to me that "change sseqs" should even exist as an option.
    change sseqs/.code={%
        \ifsseq@ispageenv\else\sseq@error@nn{change-style-data}{sseq}{#1}\fi
        \ifsseq@hasname\else\sseq@error@nn{change-style-standalone}{sseq}{#1}\fi
        \sseq@addstylecodetotoks{\sseq@sseqstyle@page}{\sseq@temppath}{#1}%
    }
}


% sseq needs special handling, so is done by hand above.
%\sseq@newstyles{/sseqpages/}{sseq}% DON'T make this /global/default! Bad things will happen.
\sseq@newstyles{/sseqpages/label,/sseqpages}{edge, label, edge label, differential label, struct line label}
\sseq@newstyles{/sseqpages/class,/sseqpages}{class, permanent cycle, transient cycle, this page cycle}
\sseq@newstyles{/sseqpages/differential,/sseqpages}{differential}
\sseq@newstyles{/sseqpages/struct line,/sseqpages}{struct line, this page struct line}
\sseq@newstyles{/sseqpages/class/label}{class label, inner class label, outer class label}
\sseq@newstyles{/sseqpages/class/label/pin}{pin}
\sseq@newstyles{/sseqpages/tikz primitives}{tikz primitive}
\sseq@newstyles{/sseqpages/fit}{circle class}


\sseq@set@newstyle{sseq/.style={sseqs={#1}},change sseq/.style={change sseqs={#1}},
    classes/.style={classs={#1}},change classes/.style={change classs={#1}},
    circle classes/.style={circle classs={#1}},change circle classes/.style={change circle classs={#1}},
}% "class" has a nonstandard plural
\sseq@set@newstyle{
    cycles/.code={\pgfkeysalso{classes={#1}}},
    permanent classes/.code={\pgfkeysalso{permanent cycles={#1}}},
    transient classes/.code={\pgfkeysalso{transient cycles={#1}}},
    this page classes/.code={\pgfkeysalso{this page cycles={#1}}},
}%

\def\sseq@newplotstyle#1{
    \sseq@setmacronospaces\sseq@tempcmd{sseq@#1style}
    \sseq@setmacronospaces\sseq@tempcmdb{sseq@#1style@default}
    \sseq@eval{\@nx\sseq@addtostorelist\@xp\@nx\sseq@tempcmd{\@xp\@nx\sseq@tempcmdb}}
    \@xp\def\sseq@tempcmdb{}
    \pgfqkeys{/sseqpages/global}{%
        #1 style/.ecode={
            \@nx\sseq@addstylecodetomacro{\@xp\@nx\sseq@tempcmd}{/tikz}{##1}
        }
    }
    % Only difference here is \sseq@tempcmdb -- which puts it in the default code -- instead of \sseq@tempcmd
    \pgfqkeys{/sseqpages/global/default}{%
        #1 style/.ecode={
            \@nx\sseq@addstylecodetomacro{\@xp\@nx\sseq@tempcmdb}{/tikz}{##1}
        }
    }
}


\def\sseq@newplotstyles#1{\def\sseq@temppath{#1}\sseq@newplotstyles@#1,\sseq@nil}
\def\sseq@newplotstyles@#1,{\sseq@newplotstyle{\sseq@trimleadingspaces#1}\@ifnextchar\sseq@nil{\@gobble}{\sseq@newplotstyles@}}

\sseq@newplotstyles{x tick, y tick}

\def\sseq@featurelist{\\{class}\\{differential}\\{structline}\\{circleclass}\\{tikzprimitive}}
\def\sseq@mapfeaturelist#1{\def\\##1{#1}\sseq@featurelist}

\def\sseq@familylist{}

\def\sseq@collections@class@hook{}
\def\sseq@collections@differential@hook{}
\def\sseq@collections@structline@hook{}
\def\sseq@collections@circleclass@hook{}
\def\sseq@collections@tikzprimitive@hook{}

\def\sseq@collections@classlabels@hook{}
\def\sseq@collections@innerclasslabels@hook{}
\def\sseq@collections@outerclasslabels@hook{}
\def\sseq@collections@pins@hook{}

\def\sseq@collections@labels@hook{}
\def\sseq@collections@edgelabels@hook{}
\def\sseq@collections@differentiallabels@hook{}
\def\sseq@collections@structlinelabels@hook{}

\def\sseq@collections@setempty#1{\sseq@collections@setempty@#1,\sseq@nil}
\def\sseq@collections@setempty@#1,{
    \pgfqkeys{/sseqpages/collections}{
        class/#1/.code={},
        differential/#1/.code={},
        struct line/#1/.code={},
        fit/#1/.code={},
        tikz primitives/#1/.code={}
    }
    \@ifnextchar\sseq@nil{\@gobble}{\sseq@collections@setempty@}
}

\sseq@collections@setempty{%
    classes, cycles, permanent classes, permanent cycles, transient classes, transient cycles, this page classes, this page cycles,
    edges, differentials,struct lines, this page struct lines,circleclasses, tikz primitives,
    labels, edge labels, differential labels, struct line labels,
    class labels, inner class labels, outer class labels,pins
}

% #1 -- path
% #2 -- hook name
% #3 -- code
\def\sseq@collections@addtohook#1#2#3{
    \sseq@keys@do{
        \let\sseq@savedoptioncode@store\sseq@savedoptioncode\let\sseq@savedoptioncode\empty
        \pgfqkeys{#1}{#3}
        \edef\sseq@savedoptioncode{\unexpanded\@xp{\sseq@savedoptioncode@store}\def\@nx#2{\unexpanded\@xp{\sseq@savedoptioncode}}}
    }
}

\pgfqkeys{/sseqpages/collections}{
    class/classes/.style={/sseqpages/class/.cd,#1},
    class/cycles/.style={/sseqpages/class/.cd,#1},
    class/permanent classes/.code={\ifsseq@permanentcycle\pgfkeysalso{/sseqpages/class/.cd,#1}\fi},
    class/permanent cycles/.code={\ifsseq@permanentcycle\pgfkeysalso{/sseqpages/class/.cd,#1}\fi},
    class/transient classes/.code={\ifsseq@permanentcycle\else\pgfkeysalso{/sseqpages/class/.cd,#1}\fi},
    class/transient cycles/.code={\ifsseq@permanentcycle\else\pgfkeysalso{/sseqpages/class/.cd,#1}\fi},
    class/this page classes/.code={\ifsseq@thispage\pgfkeysalso{/sseqpages/class/.cd,#1}\fi},
    class/this page cycles/.code={\ifsseq@thispage\pgfkeysalso{/sseqpages/class/.cd,#1}\fi},
    differential/edges/.style={/sseqpages/differential/.cd,#1},
    differential/differentials/.style={/sseqpages/differential/.cd,#1},
    struct line/edges/.style={/sseqpages/struct line/.cd,#1},
    struct line/struct lines/.style={/sseqpages/struct line/.cd,#1},
    struct line/this page struct lines/.code={\ifsseq@thispage\pgfkeysalso{/sseqpages/struct line/.cd,#1}\fi},
    fit/circleclasses/.style={/sseqpages/fit/.cd,#1},
    tikz primitives/tikz primitives/.style={/sseqpages/tikz primitives/.cd,#1},
% labels
    class/class labels/.code/.expanded={\unexpanded\@xp{\sseq@collections@addtohook{/sseqpages/class/label}{\sseq@collections@classlabels@hook}{#1}}},
    class/inner class labels/.code/.expanded={\unexpanded\@xp{\sseq@collections@addtohook{/sseqpages/class/label}{\sseq@collections@innerclasslabels@hook}{#1}}},
    class/outer class labels/.code/.expanded={\unexpanded\@xp{\sseq@collections@addtohook{/sseqpages/class/label}{\sseq@collections@outerclasslabels@hook}{#1}}},
    class/pins/.code/.expanded={\unexpanded\@xp{\sseq@collections@addtohook{/sseqpages/class/label/pin}{\sseq@collections@pins@hook}{#1}}},
%
    class/labels/.code/.expanded={\unexpanded\@xp{\sseq@collections@addtohook{/sseqpages/label}{\sseq@collections@labels@hook}{#1}}},
    differential/labels/.code/.expanded={\unexpanded\@xp{\sseq@collections@addtohook{/sseqpages/label}{\sseq@collections@labels@hook}{#1}}},
    struct line/labels/.code/.expanded={\unexpanded\@xp{\sseq@collections@addtohook{/sseqpages/label}{\sseq@collections@labels@hook}{#1}}},
%
    differential/edge labels/.code/.expanded=
        {\unexpanded\@xp{\sseq@collections@addtohook{/sseqpages/label}{\sseq@collections@edgelabels@hook}{#1}}},
    struct line/edge labels/.code/.expanded=
        {\unexpanded\@xp{\sseq@collections@addtohook{/sseqpages/label}{\sseq@collections@edgelabels@hook}{#1}}},
    differential/differential labels/.code/.expanded=
        {\unexpanded\@xp{\sseq@collections@addtohook{/sseqpages/label}{\sseq@collections@differentiallabels@hook}{#1}}},
    struct line/struct line labels/.code/.expanded=
        {\unexpanded\@xp{\sseq@collections@addtohook{/sseqpages/label}{\sseq@collections@structlinelabels@hook}{#1}}},
}

\def\sseqnewfamily#1{%
    \sseq@mapfeaturelist{
        \@xp\sseq@setmacronospaces\csname sseq@tempcmd@##1\endcsname{sseq@user@#1@##1@style}
        \@xp\sseq@setmacronospaces\csname sseq@tempcmd@##1@default\endcsname{sseq@user@#1@##1@style@default}
        \sseq@eval{\@nx\sseq@addtostorelist\@xptwo\@nx\csname sseq@tempcmd@##1\endcsname{\@xptwo\@nx\csname sseq@tempcmd@##1@default\endcsname}}
        \@xptwo\def\csname sseq@user@#1@##1@style@default\endcsname{}
    }
%
    \@xp\sseq@addtostorelist\csname sseq@#1@maybedraw\endcsname{}
    \pgfqkeys{/sseqpages/global}{%
        #1 style/.ecode={%
            \@nx\sseq@addstylecodetomacro\@xp\@nx\sseq@tempcmd@class{/sseqpages/class,/sseqpages/collections/class}{##1}%
            \@nx\sseq@addstylecodetomacro\@xp\@nx\sseq@tempcmd@differential{/sseqpages/differential,/sseqpages/collections/differential}{##1}%
            \@nx\sseq@addstylecodetomacro\@xp\@nx\sseq@tempcmd@structline{/sseqpages/struct line,/sseqpages/collections/struct line}{##1}%
            \@nx\sseq@addstylecodetomacro\@xp\@nx\sseq@tempcmd@circleclass{/sseqpages/fit,/sseqpages/collections/fit}{##1}%
            \@nx\sseq@addstylecodetomacro\@xp\@nx\sseq@tempcmd@tikzprimitive{/sseqpages/tikz primitives,/sseqpages/collections/tikz primitives}{##1}%
        },
        draw #1/.ecode = {\def\@xp\@nx\csname sseq@#1@maybedraw\endcsname{\@nx\sseq@drawtrue}},
        no #1/.ecode = {\def\@xp\@nx\csname sseq@#1@maybedraw\endcsname{\@nx\sseq@drawfalse}}
    }%
    \pgfqkeys{/sseqpages/global/default}{%
        #1 style/.ecode={%
            \@nx\sseq@addstylecodetomacro\@xp\@nx\sseq@tempcmd@class@default{/sseqpages/class}{##1}%
            \@nx\sseq@addstylecodetomacro\@xp\@nx\sseq@tempcmd@differential@default{/sseqpages/differential}{##1}%
            \@nx\sseq@addstylecodetomacro\@xp\@nx\sseq@tempcmd@structline@default{/sseqpages/struct line}{##1}%
            \@nx\sseq@addstylecodetomacro\@xp\@nx\sseq@tempcmd@circleclass@default{/sseqpages/fit}{##1}%
            \@nx\sseq@addstylecodetomacro\@xp\@nx\sseq@tempcmd@tikzprimitive@default{/sseqpages/tikz primitives}{##1}%
        },
        draw #1/.ecode = {\def\@xp\@nx\csname sseq@#1@maybedraw@default\endcsname{\@nx\sseq@drawtrue}},
        no #1/.ecode = {\def\@xp\@nx\csname sseq@#1@maybedraw@default\endcsname{\@nx\sseq@drawfalse}}
    }%
%
    \pgfqkeys{/sseqpages}{
        #1/.ecode={
            \@nx\sseq@d@addto@macro\@nx\sseq@familylist{\@nx\csname sseq@#1@maybedraw\endcsname}
            \@nx\sseq@options@secondpass{
                \@nx\sseq@d@addto@macro\@nx\sseq@collections@class@hook{\unexpanded\@xptwo{\csname sseq@tempcmd@class\endcsname}}
                \@nx\sseq@d@addto@macro\@nx\sseq@collections@differential@hook{\unexpanded\@xptwo{\csname sseq@tempcmd@differential\endcsname}}
                \@nx\sseq@d@addto@macro\@nx\sseq@collections@structline@hook{\unexpanded\@xptwo{\csname sseq@tempcmd@structline\endcsname}}
                \@nx\sseq@d@addto@macro\@nx\sseq@collections@circleclass@hook{\unexpanded\@xptwo{\csname sseq@tempcmd@circleclass\endcsname}}
                \@nx\sseq@d@addto@macro\@nx\sseq@collections@tikzprimitive@hook{\unexpanded\@xptwo{\csname sseq@tempcmd@tikzprimitive\endcsname}}
            }
        },
        class/#1/.ecode={
            \@nx\sseq@options@secondpass{\unexpanded\@xptwo{\csname sseq@tempcmd@class\endcsname}}%\showthe\@xptwo\@nx\csname sseq@tempcmd@class\endcsname
        },
        differential/#1/.code={
            \@nx\sseq@options@secondpass{\unexpanded\@xptwo{\csname sseq@tempcmd@differential\endcsname}}
        },
        struct line/#1/.code={
            \@nx\sseq@options@secondpass{\unexpanded\@xptwo{\csname sseq@tempcmd@structline\endcsname}}
        },
        fit/#1/.code={
            \@nx\sseq@options@secondpass{\unexpanded\@xptwo{\csname sseq@tempcmd@fit\endcsname}}
        },
        tikz primitives/#1/.code={
            \@nx\sseq@options@secondpass{\unexpanded\@xptwo{\csname sseq@tempcmd@tikzprimitive\endcsname}}
        }
    }
}


%%% Global keys handlers and infrastructure

\def\sseq@keys@lastpart{%
    \edef\pgfkeyscurrentkey{\pgfkeyscurrentpath}%
    \pgfkeys@split@path
    \edef\sseq@temp{\pgfkeyscurrentname}%
    \edef\sseq@temp@ii{\pgfkeyscurrentkey}%
}

\def\sseq@sanitize@intexpr#1#2{
    \sseq@ifintexpr{#1}{
        \edef\sseq@sanitize@output{\the\numexpr#1}
    }{
        \sseq@error@xx{invalid-intexpr}{\unexpanded{#2}}{#1}
        \sseq@break
    }
}

\def\sseq@sanitize@positiveintexpr#1#2{
    \sseq@sanitize@intexpr{#1}{#2}
    \ifnum\sseq@sanitize@output<1\relax
        \sseq@error@xx{invalid-positive-intexpr}{\unexpanded{#2}}{#1}
        \@xp\sseq@break
    \fi
}

\def\sseq@sanitize@dimenexpr#1#2{
    \sseq@ifpgfmathexpr{#1}{
        \ifpgfmathunitsdeclared\else
            \sseq@error@xx{invalid-dimenexpr}{\unexpanded{#2}}{#1}
            \@xp\sseq@break
        \fi
        \edef\sseq@sanitize@output{\sseq@mathresult}
    }{
        \sseq@error@xx{invalid-dimenexpr}{\unexpanded{#2}}{#1}
        \sseq@break
    }
}

\def\sseq@sanitize@color#1#2{
    \begingroup
        \global\sseq@gtempiffalse
        \def\c@lor@error##1{\global\sseq@gtempiftrue}
        \color{#1}
    \endgroup
    \ifsseq@gtempif
        \sseq@error@xx{invalid-color}{\unexpanded{#2}}{#1}
        \@xp\sseq@break
    \else
        \def\sseq@sanitize@output{#1}
    \fi
}

\pgfqkeys{/handlers}{
    .sseq store in/.code 2 args={%
        \sseq@addtostorelist#1{#2}%
        \sseq@keys@lastpart
        \pgfkeys{%
            \sseq@temp@ii/.store in=#1,%
            /sseqpages/global/default/\sseq@temp/.store in/.expand once=\csname \sseq@macroname#1@default\endcsname
        }%
    },
    .sseq store in sanitize/.code n args={3}{%
        \sseq@addtostorelist#1{#2}%
        \sseq@keys@lastpart
        \pgfkeys{%
            \sseq@temp@ii/.code/.expanded={%
                \unexpanded{#3{##1}}{\sseq@temp=\unexpanded{##1}}%
                \unexpanded{\edef#1{\unexpanded\@xp{\sseq@sanitize@output}}}%
                %\show\@nx#1
                \sseq@breakpoint
            },
            \sseq@temp@ii/.value required,
            /sseqpages/global/default/\sseq@temp/.code/.expanded={%
                \unexpanded{#3{##1}}{\sseq@temp=\unexpanded{##1}}%
                \edef\@xp\@nx\csname \sseq@macroname#1@default\endcsname\unexpanded{{\unexpanded\@xp{\sseq@sanitize@output}}}%
                \sseq@breakpoint
            },
            /sseqpages/global/default/\sseq@temp/.value required
        }%

    },
    .sseq store int in/.code 2 args={
        \sseq@keys@lastpart
        \pgfkeysalso{%
            \sseq@temp@ii/.sseq store in sanitize={#1}{#2}\sseq@sanitize@intexpr
        }
    },
    .sseq store positive int in/.code 2 args={
        \sseq@keys@lastpart
        \pgfkeysalso{%
            \sseq@temp@ii/.sseq store in sanitize={#1}{#2}\sseq@sanitize@positiveintexpr
        }
    },
    .sseq store dimen in/.code 2 args={
        \sseq@keys@lastpart
        \pgfkeysalso{%
            \sseq@temp@ii/.sseq store in sanitize={#1}{#2}\sseq@sanitize@dimenexpr
        }
    },
    .sseq append to style/.code 2 args={%
        \sseq@addkeytostorelist{#1}{\pgfkeysalso{#2}}%
        \sseq@keys@lastpart
        \pgfkeys{%
            \sseq@temp@ii/.style={#1/.append style={##1}},%
            /sseqpages/global/default/\sseq@temp/.ecode={%
                \@nx\apptocmd\@xp\@nx\csname pgfk@#1/.@cmd@default\endcsname{\@nx\pgfkeysalso{##1}}{}{\@nx\sseq@error{this-shouldnt-happen}}%
                \global\let\@xp\@nx\csname pgfk@#1/.@cmd@default\endcsname\@xp\@nx\csname pgfk@#1/.@cmd@default\endcsname
            }%
        }%
    },
    .sseq is if/.code={%
        \sseq@addiftostorelist@truedefault{#1}%
        \sseq@keys@lastpart
        \pgfkeys{%
            \sseq@temp@ii/.is if=sseq@#1,%
            /sseqpages/global/default/\sseq@temp/.code={\@xp\edef\csname sseq@#1@default\endcsname{\global\@xp\@nx\csname sseq@#1##1\endcsname}},%
            /sseqpages/global/default/\sseq@temp/.default=true,%
        }%
    },
    .sseq is if false/.code={%
        \sseq@keys@lastpart
        \pgfkeys{%
            /sseqpages/global/\sseq@temp/.code/.expanded={\@xp\@nx\csname sseq@#1false\endcsname},
            /sseqpages/global/\sseq@temp/.value forbidden,
            % value forbidden and ecode don't work correctly together so we have to use .code/.expanded .
            % I made a thread here: https://sourceforge.net/p/pgf/discussion/477362/thread/3e75d813/
            /sseqpages/global/default/\sseq@temp/.code/.expanded={\def\@xp\@nx\csname sseq@#1@default\endcsname{\global\@xp\@nx\csname sseq@#1false\endcsname}},
            /sseqpages/global/default/\sseq@temp/.value forbidden,
        }%
    },
    .sseq default code/.code={%
        \sseq@keys@lastpart
        \pgfkeys{/sseqpages/global/default/\sseq@temp/.code={#1}}%
    },
    .sseq default code 2 args/.code={%
        \sseq@keys@lastpart
        \pgfkeys{/sseqpages/global/default/\sseq@temp/.code 2 args={#1}}%
    },
    .sseq default error/.code={%
        \sseq@keys@lastpart
        \pgfkeys{/sseqpages/global/default/\sseq@temp/.ecode={\@nx\sseq@error@n{option-no-default-allowed}{\sseq@temp}}}%
    },
    .sseq x y alias/.code={%
        \sseq@keys@lastpart
        \ifx\pgfkeysnovalue#1%
            \edef\sseq@tempb##1{##1 \sseq@temp}%
        \else
            \def\sseq@tempb##1{#1}%
            \sseq@usedparam\sseq@tempb
        \fi
        \pgfkeys{%
            /sseqpages/global/\sseq@temp/.ecode={\@nx\pgfkeysalso{\sseq@tempb{x}={##1},\sseq@tempb{y}={##1}}},%
            /sseqpages/global/default/\sseq@temp/.ecode={\@nx\pgfkeysalso{\sseq@tempb{x}={##1},\sseq@tempb{y}={##1}}}%
        }%
    }
}

\bgroup\lccode`*=`#\lowercase{\egroup
    \def\sseq@usedparam#1{
        \@xp\sseq@usedparam@\@xp#1\meaning#1*\@nil
    }
    \def\sseq@usedparam@#1#2->#3*#4\@nil{
        \sseq@ifempty{#4}{
            \@xp\def\@xp#1\@xp##\@xp1\@xp{\@xp##\@xp1\@xp\space#1{}}
        }{}
    }
}

\def\sseq@patchrepeat#1#2#3{%
    \patchcmd#1{#2}{#3}{\sseq@patchrepeat{#1}{#2}{#3}}{}%
}

\def\sseq@stylehandler@processpath#1#2{
    \def\sseq@savedoptioncode{}%
    \sseq@eval{\@nx\sseq@addstylecodetooptions{/sseqpages/#1}{\unexpanded\@xp{\sseq@keybody{*1}}}}%
    \edef\sseq@savedoptioncode##1{\unexpanded\@xp{\sseq@savedoptioncode}}
    \sseq@patchrepeat\sseq@savedoptioncode{*1}{##1}
    \sseq@eval{\@nx\pgfqkeys{/sseqpages/#1}{\sseq@temp/.#2={\unexpanded\@xp{\sseq@savedoptioncode{##1}}}}}%
}

\def\sseq@add@tikzkey@to@options@forstylehander#1{\sseq@e@addto@options{#1}}
\pgfqkeys{/handlers}{ % defines sseq style, etc
    sseq @@ make style handler/.style 2 args={%
        .sseq #1/.code={%
            \def\sseq@keybody####1{##1}
            \sseq@keys@lastpart
            \let\sseq@savedoptioncode@store\sseq@savedoptioncode
            \def\sseq@keys@unknown@error{}%
            \let\sseq@add@tikzkey@to@options\sseq@add@tikzkey@to@options@forstylehander
%
            \sseq@stylehandler@processpath{scope}{#2}%
            %\tracingall
            \sseq@stylehandler@processpath{class}{#2}
            \sseq@stylehandler@processpath{differential}{#2}%
            \sseq@stylehandler@processpath{struct line}{#2}%
            \sseq@stylehandler@processpath{circle classes}{#2}%
            \sseq@stylehandler@processpath{tikz primitives}{#2}%
%
            \let\sseq@savedoptioncode\sseq@savedoptioncode@store
            \let\sseq@addstylecodetooptions@unknown\sseq@keys@unknown
            \let\sseq@keys@unknown@error\sseq@keys@unknown@error@normal
            \let\sseq@add@tikzkey@to@options\sseq@e@addto@options
        }%
    },%
    sseq @@ make code handler/.style={%
        .sseq #1/.code={%
            \sseq@keys@lastpart
            \pgfkeys{
                /sseqpages/class/\sseq@temp/.#1={##1},%
                /sseqpages/differential/\sseq@temp/.#1={##1},%
                /sseqpages/struct line/\sseq@temp/.#1={##1}%
            }%
        }%
    },
%
    sseq @@ make style handler={style}{code},
    sseq @@ make style handler={prefix style}{prefix code},
    sseq @@ make style handler={append style}{append code},
    sseq @@ make code handler=code,
    sseq @@ make code handler=prefix code,
    sseq @@ make code handler=append code,
% these just are the same as /.style, etc.
    .global sseq style/.code={\pgfkeys{\pgfkeyscurrentpath/.code=\pgfkeysalso{#1}}},
    .global sseq prefix style/.code={\pgfkeys{\pgfkeyscurrentpath/.add code={\pgfkeysalso{#1}}{}}},
    .global sseq append style/.code={\pgfkeys{\pgfkeyscurrentpath/.add code={}{\pgfkeysalso{#1}}}},
%
%
    .sseq grading/.code 2 args={%
        \sseq@keys@lastpart
        \pgfkeys{%
            /sseqpages/global/\sseq@temp/.estyle={\unexpanded{degree={#1}{#2}}},%I'm not really sure why these need to be estyle and \unexpanded but it does...
            /sseqpages/global/default/\sseq@temp/.estyle={\unexpanded{degree={#1}{#2}}},%
            /sseqpages/\sseq@temp/.sseq @ global only
        }%
    },
    Adams grading/.sseq grading={-1}{#1},
    homological Serre grading/.sseq grading={-#1}{#1-1},
    cohomological Serre grading/.sseq grading={#1}{1-#1},
}

%%% Define global keys setter
\gdef\sseq@globalkeys#1{%
    \let\sseq@storetilde~%
    \def~{}%
    \pgfqkeys{/sseqpages/global}{#1}%
    \sseq@globalkeys@#1,\sseq@nil
    \let~\sseq@storetilde
}

\def\sseq@globalkeys@#1,{%
    \sseq@tempiftrue
    \sseq@keys@removehandler{#1}%
    \ifsseq@tempif
        \pgfkeys{/sseqpages/\sseq@temp/.sseq @ global only}%
        \pgfkeysifdefined{/sseqpages/global/default/\sseq@temp/.@cmd}{}{%
            \pgfkeys{/sseqpages/global/default/\sseq@temp/.ecode={\@nx\pgfkeysalso{sseqs={\sseq@temp}}}}%
        }%
    \fi
    \sseq@globalkeys@next
}

\def\sseq@globalkeys@eatentry#1,{%
    \sseq@globalkeys@next
}

\def\sseq@globalkeys@next{%
    \@ifnextchar\sseq@nil{\@gobble}{%
        \@ifnextchar~{\sseq@globalkeys@eatentry}{%
            \@ifnextchar,{\sseq@globalkeys@eatentry}{%
                \sseq@globalkeys@
            }%
        }%
    }%
}

\def\sseq@keys@ignorehandlers{%
    \\{value forbidden}\\{value required}\\{default}%\\{sseq store in}\\{sseq is if}%
    \\{sseq default code}\\{sseq default code 2 args}%
}
\def\sseq@keys@removehandler#1{\sseq@keys@removehandler@#1=\sseq@nil}
\def\sseq@keys@removehandler@#1/.#2=#3\sseq@nil{%
    \def\sseq@temp{#2}%
    \def\\##1{\def\sseq@tempii{##1}\ifx\sseq@tempii\sseq@temp \sseq@tempiffalse\let\\\@gobble\fi}%
    \sseq@keys@ignorehandlers
    \def\sseq@temp{#1}%
}


%%% Define global keys
\def\sseq@currentoptionpath{default}
\pgfkeys{
    % \sseq@currentoptionpath is set to \sseq@thename in \sseqdata, then "temp" in \sseqpage
    /sseqpages/global/.unknown/.code={%
        \let\sseq@tempa\pgfkeyscurrentname
        \let\sseq@tempb\pgfkeyscurrentvalue
        \sseq@eval{\@nx\pgfqkeys{/sseqpages/global}{/sseqpages/userstyles/\sseq@tempa/.try={\unexpanded\@xp{\sseq@tempb}}}}
        \ifpgfkeyssuccess\else
            \pgfkeysifdefined{/tikz/\pgfkeyscurrentname/.@cmd}{\@firstofone}{\pgfkeysifdefined{/tikz/\pgfkeyscurrentname}{\@firstofone}{%
                    \sseq@error{unknown-global}\@gobble
            }}{%
                \pgfkeysalso{{\sseq@currentoptionpath} options/.append code/.expand twice={\@xp\@xp\@xp\tikzset\@xp\@xp\@xp{\@xp\sseq@tempa\@xp=\@xp{\sseq@tempb}}}}%
                \sseq@eval{%
                    \global\let\@xp\@nx\csname pgfk@/sseqpages/global/{\sseq@currentoptionpath} options/.@cmd\endcsname
                        \@xp\@nx\csname pgfk@/sseqpages/global/{\sseq@currentoptionpath} options/.@cmd\endcsname
                }%
            }%
        \fi
    },
    /sseqpages/global/default/.unknown/.code={%
        \sseq@passoptionto{/sseqpages/global}%
    },
    /sseqpages/global/{default} options/.code={},
    /sseqpages/global/name only/.unknown/.code={},
    /sseqpages/global/sseqstyle/.unknown/.code={}
}

\pgfqkeys{/sseqpages/global/name only}{
    name/.code={\def\sseq@thename{#1}\sseq@hasnametrue},
    name/.value required,
    keep changes/.is if=sseq@keepchanges,
    update existing/.code={\ifsseq@ispageenv\sseq@error{page-update-existing}\fi\sseq@updateexistingtrue},
    update existing/.sseq default code={\sseq@updateexistingtrue}
}

\sseq@addtostorelist\sseq@grid{\relax}
\sseq@addiftostorelist{hasdegree}
\let\sseq@maybemathswitch\relax
\sseq@globalkeys{%
    name/.code={},
    name/.sseq default error,
    ~page/.code={
        \ifsseq@hasname
            \sseq@getnumrange\sseq@thepage{#1}
            \sseq@thepagecount=\sseq@thepage\relax
            \let\page\sseq@thepage
            \edef\sseq@thepagemm{\the\numexpr\sseq@thepagecount-1\relax}
            \ifx\sseq@thepagemax\undefined
                \edef\sseq@thepagemaxpp{\the\numexpr\sseq@thepagecount+1\relax}
            \else
                \ifnum\sseq@thepagemax=\z@
                    \let\sseq@thepagemaxpp\sseq@infinity
                \else
                    \edef\sseq@thepagemaxpp{\the\numexpr\sseq@thepagemax+1\relax}
                \fi
            \fi
        \else
            \sseq@error{page-no-name}
        \fi
    },
    ~page/.sseq default error,% implement page default?
    degree/.code 2 args={\def\sseq@targetx##1{#1}\def\sseq@targety##1{#2}},
    degree/.sseq default code 2 args={\def\sseq@targetx@default##1{#1}\def\sseq@targety@default##1{#2}},
    strict degree/.sseq is if=strictdegree,
    lax degree/.sseq is if false=strictdegree,
    update existing/.code={}, % processed with "name" key and "keep changes" key in /sseqpages/global/name only
    keep changes/.code={%
        \ifsseq@hasname\else\sseq@error{keep-changes-no-name}\fi
        \ifsseq@ispageenv\else\sseq@error{keep-changes-sseqdata}\fi
    },
    grid/.code={\ifcsname sseq@grid@#1\endcsname\@xp\let\@xp\sseq@grid\csname sseq@grid@#1\endcsname\else\sseq@error@n{unknown-grid}{#1}\fi},
    grid/.sseq default code={\ifcsname sseq@grid@#1\endcsname\@xp\let\@xp\sseq@grid@default\csname sseq@grid@#1\endcsname\else\sseq@error@n{unknown-grid}{#1}\fi},
    grid color/.sseq store in sanitize=\sseq@gridcolor{gray}\sseq@sanitize@color,
    grid step/.sseq x y alias,
    grid step/.value required,
    x grid step/.sseq store positive int in=\sseq@xgridstep{1},
    y grid step/.sseq store positive int in=\sseq@ygridstep{1},
    math nodes/.sseq is if=mathnodes, % TODO: make this locally toggleable
    custom clip/.sseq store in=\sseq@customclip{},
    clip/.sseq is if=clip,
    draw differentials/.sseq is if=drawdifferentials,
    no differentials/.sseq is if false={drawdifferentials},
    draw struct lines/.sseq is if=drawstructlines,
    no struct lines/.sseq is if false={drawstructlines},
    draw orphan edges/.sseq is if=draworphanedges,
    no orphan edges/.sseq is if false={draworphanedges},
    class pattern/.sseq store in=\sseq@classpattern{standard},
    class pattern/.add code={\ifcsname sseq@#1xoffset1/1\endcsname\else\sseq@error{unknown-class-pattern}\@xp\sseq@break\fi}{\sseq@breakpoint},
    class placement transform/.code={%
        \bgroup
        \let\tikz@transform\pgfutil@empty
        \pgfqkeys{/sseqpages/class placement}{#1}%
        \global\let\sseq@classplacementtransform\tikz@transform
        \egroup
    },
    class placement transform/.sseq default code={%
        \bgroup
        \let\tikz@transform\pgfutil@empty
        \pgfqkeys{/sseqpages/class placement}{#1}%
        \global\let\sseq@classplacementtransform@default\tikz@transform
        \egroup
    },
    class placement transform/.value required,
    add class placement transform/.code={%
        \bgroup
        \let\tikz@transform\pgfutil@empty
        \pgfqkeys{/sseqpages/class placement}{#1}%
        \sseq@x@addto@macro\sseq@classplacementtransform{\unexpanded\@xp\tikz@transform}
        \egroup
    },
    add class placement transform/.sseq default code={%
        \bgroup
        \let\tikz@transform\pgfutil@empty
        \pgfqkeys{/sseqpages/class placement}{#1}%
        \sseq@x@addto@macro\sseq@classplacementtransform@default{\unexpanded\@xp\tikz@transform}
        \egroup
    },
    add class placement transform/.value required,
}

\sseq@mathnodestrue

\sseq@addtostorelist\sseq@targetx{}\sseq@addtostorelist\sseq@targety{}
\sseq@addtostorelist\sseq@xtickstep{1}\sseq@addtostorelist\sseq@ytickstep{1}
\sseq@addtostorelist\sseq@classplacementtransform{}



%%% Range keys and commands
\sseq@globalkeys{%
    x range/.code 2 args={\sseq@xrange{#1}{#2}},
    x range/.sseq default code 2 args={\sseq@xrange@default{#1}{#2}},
    y range/.code 2 args={\sseq@yrange{#1}{#2}},
    y range/.sseq default code 2 args={\sseq@yrange@default{#1}{#2}},
    xrange/.forward to=/sseqpages/global/x range,
    yrange/.forward to=/sseqpages/global/y range
}

% These are set up so that if the user specifies no \xrange, the minimum position of any node can be calculated on the fly.
\@xp\sseq@addtostorelist\@xp\sseq@xmin\@xp{\sseq@infinity}
\@xp\sseq@addtostorelist\@xp\sseq@ymin\@xp{\sseq@infinity}
\@xp\sseq@addtostorelist\@xp\sseq@xmax\@xp{\@xp-\sseq@infinity}
\@xp\sseq@addtostorelist\@xp\sseq@ymax\@xp{\@xp-\sseq@infinity}
\sseq@addtostorelist\xmin{\protect\sseq@xmin}
\sseq@addtostorelist\ymin{\protect\sseq@ymin}
\sseq@addtostorelist\xmax{\protect\sseq@xmax}
\sseq@addtostorelist\ymax{\protect\sseq@ymax}
% These are used to decide whether to add a certain node to savedpaths.
% In general the check is whether xmin<=xcoord<=xmax, but since Latex has no <=, set xminmm=xmin-1, xmaxpp=xmax+1.
% If the user doesn't specify a range, all nodes should be rendered
% To ensure that xminmm<xccord<xminpp in that case, set the default to xminmm=-\infty, xmaxpp=+\infty.
\@xp\sseq@addtostorelist\@xp\sseq@xminmm\@xp{\@xp-\sseq@infinity}
\@xp\sseq@addtostorelist\@xp\sseq@yminmm\@xp{\@xp-\sseq@infinity}
\@xp\sseq@addtostorelist\@xp\sseq@xmaxpp\@xp{\sseq@infinity}
\@xp\sseq@addtostorelist\@xp\sseq@ymaxpp\@xp{\sseq@infinity}

\sseq@addiftostorelist{specifiedxrange}
\sseq@addiftostorelist{specifiedyrange}
\def\sseq@xrange#1#2{%
    \global\let\sseq@xminmax\@gobbletwo % Don't need to calculate range of nodes if user specifies it
    \global\sseq@specifiedxrangetrue
    \sseq@sanitize@intexpr{#1}{x range={#1}{#2}}
    \sseq@sanitize@intexpr{#2}{x range={#1}{#2}}
    \xdef\sseq@xmin{\the\numexpr#1}%
    \xdef\sseq@xmax{\the\numexpr#2}%
    \let\xmin\sseq@xmin
    \let\xmax\sseq@xmax
    % Need to know decremented \xmin and incremented \xmax to decide which nodes to print
    \sseq@tempcount=#1 \advance\sseq@tempcount-1\xdef\sseq@xminmm{\the\sseq@tempcount}%
    \sseq@tempcount=#2 \advance\sseq@tempcount1\xdef\sseq@xmaxpp{\the\sseq@tempcount}%
    \sseq@breakpoint
}
\def\sseq@yrange#1#2{%
    \global\let\sseq@yminmax\@gobbletwo %
    \sseq@sanitize@intexpr{#1}{x range={#1}{#2}}
    \sseq@sanitize@intexpr{#2}{x range={#1}{#2}}
    \global\sseq@specifiedyrangetrue
    \gdef\sseq@ymin{#1}%
    \gdef\sseq@ymax{#2}%
    \let\ymin\sseq@ymin
    \let\ymax\sseq@ymax
    \sseq@tempcount=#1 \advance\sseq@tempcount-1\xdef\sseq@yminmm{\the\sseq@tempcount}%
    \sseq@tempcount=#2 \advance\sseq@tempcount1\xdef\sseq@ymaxpp{\the\sseq@tempcount}%
}

\def\sseq@xrange@default#1#2{%
    \global\let\sseq@xminmax\@gobbletwo % Don't need to calculate range of nodes if user specifies it
    \gdef\sseq@xmin@default{#1}%
    \gdef\sseq@xmax@default{#2}%
    % Need to know decremented \xmin and incremented \xmax to decide which nodes to print
    \sseq@tempcount=#1 \advance\sseq@tempcount-1\xdef\sseq@xminmm@default{\the\sseq@tempcount}%
    \sseq@tempcount=#2 \advance\sseq@tempcount1\xdef\sseq@xmaxpp@default{\the\sseq@tempcount}%
}
\def\sseq@yrange@default#1#2{%
    \global\let\sseq@yminmax\@gobbletwo %
    \gdef\sseq@ymin@default{#1}%
    \gdef\sseq@ymax@default{#2}%
    \sseq@tempcount=#1 \advance\sseq@tempcount-1\xdef\sseq@yminmm@default{\the\sseq@tempcount}%
    \sseq@tempcount=#2 \advance\sseq@tempcount1\xdef\sseq@ymaxpp@default{\the\sseq@tempcount}%
}

% If #1 is less than \xmin or greater than \xmax, update \xmin or \xmax appropriately
% Gets set to \@gobble if user specifies range
\def\sseq@xminmax@active#1#2{%
    \ifnum#1<\sseq@xmin\relax
        \ifx\sseq@yminmax\@gobbletwo
            \ifnum#2>\sseq@yminmm\relax
                \ifnum#2<\sseq@ymaxpp\relax
                    \xdef\sseq@xmin{#1}%
                \fi
            \fi
        \else
            \xdef\sseq@xmin{#1}%
        \fi
    \fi
    \ifnum#1>\sseq@xmax\relax
        \ifx\sseq@yminmax\@gobbletwo
            \ifnum#2>\sseq@yminmm\relax
                \ifnum#2<\sseq@ymaxpp\relax
                    \xdef\sseq@xmax{#1}%
                \fi
            \fi
        \else
            \xdef\sseq@xmax{#1}%
        \fi
    \fi
}

\def\sseq@yminmax@active#1#2{%
    \ifnum#2<\sseq@ymin\relax
        \ifx\sseq@xminmax\@gobbletwo
            \ifnum#1>\sseq@xminmm\relax
                \ifnum#1<\sseq@xmaxpp\relax
                    \xdef\sseq@ymin{#2}%
                \fi
            \fi
        \else
            \xdef\sseq@ymin{#2}%
        \fi
    \fi
    \ifnum#2>\sseq@ymax\relax
        \ifx\sseq@xminmax\@gobbletwo
            \ifnum#1>\sseq@xminmm\relax
                \ifnum#1<\sseq@xmaxpp\relax
                    \xdef\sseq@ymax{#2}%
                \fi
            \fi
        \else
            \xdef\sseq@ymax{#2}%
        \fi
    \fi
}

%%% Axes and Layout options
\sseq@addtostorelist{\sseq@xtickfn}{} % These should initially be the identity macro, but addtostorelist doesn't accept functions as default
\sseq@addtostorelist{\sseq@ytickfn}{} % so the default is instated in begin{sseqdata} / begin{sseqpicture}.

\sseq@globalkeys{%
    title/.sseq store in=\sseq@title{},
	title style/.sseq append to style={/sseqpages/global/every title}{at={(0.5*\xmin+0.5*\xmax,\ymax)}},
    draw title/.sseq is if=drawtitle,
    no title/.code=\sseq@drawtitlefalse,
%
    x label/.sseq store in=\sseq@xaxislabel{},
    y label/.sseq store in=\sseq@yaxislabel{},
	x label style/.sseq append to style={/sseqpages/global/every x axis label}{at={(0.5*\xmin+0.5*\xmax,\ymin)}},
	y label style/.sseq append to style={/sseqpages/global/every y axis label}{at={(\xmin ,0.5*\ymax+0.5*\ymin)}},
	label style/.sseq x y alias,
    draw x label/.sseq is if=drawxaxislabel,
    draw y label/.sseq is if=drawyaxislabel,
    no x label/.sseq is if false={drawxaxislabel},
    no y label/.sseq is if false={drawyaxislabel},
    no labels/.sseq x y alias=no #1 label,
    draw labels/.sseq x y alias=draw #1 label,
%
    no x axis/.sseq is if false={drawxaxis},
    no y axis/.sseq is if false={drawyaxis},
    no axes/.value forbidden,
    no axes/.sseq x y alias=no #1 axis,
    draw x axis/.sseq is if=drawxaxis,
    draw y axis/.sseq is if=drawyaxis,
    draw axes/.sseq x y alias=draw #1 axis,
    rotate labels/.sseq is if=rotatelabels,
%
    draw x ticks/.sseq is if=drawxaxisticks,
    no x ticks/.sseq is if false=drawxaxisticks,
    draw y ticks/.sseq is if=drawyaxisticks,
    no y ticks/.sseq is if false=drawyaxisticks,
    no ticks/.sseq x y alias=no #1 ticks,
    no ticks/.value forbidden,
    draw ticks/.sseq x y alias=draw #1 ticks,
    draw ticks/.default=true,
    tick step/.sseq x y alias,
    tick step/.value required,
    x tick step/.sseq store positive int in=\sseq@xtickstep{1},
    y tick step/.sseq store positive int in=\sseq@ytickstep{1},
    tick offset/.sseq x y alias,
    tick offset/.value required,
    x tick offset/.sseq store int in=\sseq@xtickstepoffset{0},
    y tick offset/.sseq store int in=\sseq@ytickstepoffset{0},
    x tick handler/.code={\def\sseq@xtickfn##1{#1}},
    x tick handler/.sseq default code={\def\sseq@xtickfn@default##1{#1}},
    y tick handler/.code={\def\sseq@ytickfn##1{#1}},
    y tick handler/.sseq default code={\def\sseq@ytickfn@default##1{#1}},
    class label handler/.code = { % \long so that \ifx\classlabeltextfn\@gobble compares true if this is empty (\@gobble is long...)
        \long\def\sseq@classlabeltextfn##1{#1}
        \ifx\sseq@classlabeltextfn\@gobble
            \def\sseq@classlabeltextfn##1{\def\processedlabel{##1}}
        \else
            \edef\sseq@tempa{\unexpanded\@xp{\sseq@classlabeltextfn{}}}
            \edef\sseq@tempb{\unexpanded\@xp{\sseq@classlabeltextfn{blah}}}
            \ifx\sseq@tempa\sseq@tempb
                \long\def\sseq@classlabeltextfn##1{#1{##1}}
            \fi
        \fi
    },
    class label handler/.sseq default code={
        \long\def\sseq@classlabeltextfn@default##1{#1}
        \ifx\sseq@classlabeltextfn@default\@gobble
            \def\sseq@classlabeltextfn@default##1{\def\processedlabel{##1}}
        \else
            \edef\sseq@tempa{\unexpanded\@xp{\sseq@classlabeltextfn@default{}}}
            \edef\sseq@tempb{\unexpanded\@xp{\sseq@classlabeltextfn@default{blah}}}
            \ifx\sseq@tempa\sseq@tempb
                \long\def\sseq@classlabeltextfn@default##1{#1{##1}}
            \fi
        \fi
    },
    x tick style/.sseq store in=\sseq@xtickstyle{},
    y tick style/.sseq store in=\sseq@ytickstyle{},
    tick style/.sseq x y alias,
%
    axes type/.sseq x y alias=axis type,
    x axis type/.sseq store in=\sseq@xaxistype{border},
    y axis type/.sseq store in=\sseq@yaxistype{border},
    x axis origin/.sseq store int in=\sseq@xaxisorigin@center@{0},
    y axis origin/.sseq store int in=\sseq@yaxisorigin@center@{0},
    axes origin/.sseq x y alias=axis origin,
}


\pgfqkeys{/sseqpages}{%
    class label handler/.code = { % \long so that \ifx\classlabeltextfn\@gobble compares true if this is empty (\@gobble is long...)
        \long\def\sseq@classlabeltextfn##1{#1}
        \ifx\sseq@classlabeltextfn\@gobble
            \def\sseq@classlabeltextfn##1{\def\processedlabel{##1}}
        \else
            \edef\sseq@tempa{\unexpanded\@xp{\sseq@classlabeltextfn{}}}
            \edef\sseq@tempb{\unexpanded\@xp{\sseq@classlabeltextfn{blah}}}
            \ifx\sseq@tempa\sseq@tempb
                \long\def\sseq@classlabeltextfn##1{#1{##1}}
            \fi
        \fi
    }
}

\sseq@globalkeys{
    axes gap/.sseq x y alias=axis gap,
    axes gap/.value required,
    x axis gap/.sseq store dimen in=\sseq@xaxisgap{0.5cm},
    y axis gap/.sseq store dimen in=\sseq@yaxisgap{0.5cm},
    tick gap/.sseq x y alias,
    tick gap/.value required,
    x tick gap/.sseq store dimen in=\sseq@xlabelgap{0.5cm},
    y tick gap/.sseq store dimen in=\sseq@ylabelgap{0.5cm},
    x axis extend start/.sseq store dimen in=\sseq@xaxis@start@extend{0.9cm},
    y axis extend start/.sseq store dimen in=\sseq@yaxis@start@extend{0.9cm},
    axes extend start/.sseq x y alias=axis extend start,
    x axis extend end/.sseq store dimen in=\sseq@xaxis@end@extend{0.5cm},
    y axis extend end/.sseq store dimen in=\sseq@yaxis@end@extend{0.5cm},
    axes extend end/.sseq x y alias=axis extend end,
    x axis clip padding/.sseq store dimen in=\sseq@xclip@axisgap{0.1cm},
    y axis clip padding/.sseq store dimen in=\sseq@yclip@axisgap{0.1cm},
    axes clip padding/.sseq x y alias=axis clip padding,
    right clip padding/.sseq store dimen in=\sseq@clip@padding@right{0.1cm},
    left clip padding/.sseq store dimen in=\sseq@clip@padding@left{0.4cm},
    top clip padding/.sseq store dimen in=\sseq@clip@padding@top{0.1cm},
    bottom clip padding/.sseq store dimen in=\sseq@clip@padding@bottom{0.4cm}
}

%%%
%%% Coordinate transformations
%%%

%%% Global Coordinate Transformations
\sseq@globaldetonetrue
\sseq@addtostorelist\sseq@globaldetonetransform{}
\sseq@addtostorelist\sseq@globalrotatetransform{}
\sseq@addtostorelist\sseq@globalscaletransform{}

\sseq@addtostorelist\sseq@xscale{1}
\sseq@addtostorelist\sseq@yscale{1}

\def\sseq@globaldetoneoption#1#2{%
    \ifsseq@globaldetone\else
        \sseq@error@n{det-one-after-scale}{#2}
    \fi
    \@xp\gdef\@xp\sseq@globaldetonetransform\@xp{\sseq@globaldetonetransform #1}%
    #1
}
\def\sseq@globalscaleoption#1{%
    \sseq@globaldetonefalse
    \@xp\gdef\@xp\sseq@globalscaletransform\@xp{\sseq@globalscaletransform#1}%
}
\pgfqkeys{/sseqpages/global}{%
    x/.sseq @ illegal,
    y/.sseq @ illegal,
    z/.sseq @ illegal,
    xshift/.sseq @ illegal as global,
    yshift/.sseq @ illegal as global,
    shift/.sseq @ illegal as global,
    scale around/.sseq @ illegal as global,
    rotate around/.sseq @ illegal as global,
    rotate/.code={%
        \@xp\gdef\@xp\sseq@globalrotatetransform\@xp{\sseq@globalrotatetransform\pgftransformrotate{#1}}%
        \sseq@globaldetoneoption{\pgftransformrotate{#1}}{rotate}%
    },
    rotate/.value required,
    xslant/.code={\sseq@globaldetoneoption{\pgftransformxslant{#1}}{xslant}},
    xslant/.value required,
    yslant/.code={\sseq@globaldetoneoption{\pgftransformyslant{#1}}{yslant}},
    yslant/.value required,
    xmirror/.code={\sseq@globaldetoneoption{\pgftransformxscale{-1}}{xmirror}},
    xmirror/.value forbidden,
    ymirror/.code={\sseq@globaldetoneoption{\pgftransformyscale{-1}}{ymirror}},
    ymirror/.value forbidden,
    mirror/.forward to=/sseqpages/global/xmirror,
    mirror/.forward to=/sseqpages/global/ymirror,
    xscale/.code={%
        \sseq@ifpgfmathexpr{int(#1-1)}{}{
            \sseq@error@nn{invalid-scale}{x}{#1}%
            \sseq@break
        }
        \ifpgfmathunitsdeclared
            \sseq@error@nn{unit-scale}{x}{#1}%
            \sseq@break
        \fi
        \ifnum\sseq@mathresult<\z@
            \sseq@error@nn{negative-scale}{x}{#1}%
            \sseq@break
        \fi
        \sseq@globalscaleoption{\pgfsetxvec{\pgfpointxy{#1}{0}}}
        \pgfmathparse{\sseq@xscale*#1}
        \edef\sseq@xscale{\pgfmathresult}
        \sseq@breakpoint
    },
    xscale/.value required,
    yscale/.code={%
        \sseq@ifpgfmathexpr{int(#1-1)}{}{
            \sseq@error@nn{invalid-scale}{y}{#1}%
            \sseq@break
        }
        \ifpgfmathunitsdeclared
            \sseq@error@nn{unit-scale}{y}{#1}%
            \sseq@break
        \fi
        \ifnum\sseq@mathresult<\z@
            \sseq@error@nn{negative-scale}{y}{#1}%
            \sseq@break
        \fi
        \sseq@globalscaleoption{\pgfsetyvec{\pgfpointxy{0}{#1}}}
        \pgfmathparse{\sseq@yscale*#1}
        \edef\sseq@yscale{\pgfmathresult}
        \sseq@breakpoint
    },
    yscale/.value required,
    scale/.sseq x y alias=#1scale,
    scale/.value required
}

%%% Local Coordinate transformations

\pgfqkeys{/sseqpages/class placement}{
    x shift/.code={\sseq@error@nn{class-placement-shift}{x shift}{#1}},
    y shift/.code={\sseq@error@nn{class-placement-shift}{y shift}{#1}},
    shift/.code={\sseq@error@nn{class-placement-shift}{shift}{#1}}
}

%% setup for coordinate transformation keys
\def\sseq@checktransform{\sseq@error@nn{illegal-transform}}
\pgfqkeys{/sseqpages}{%
    x/.code={\sseq@error@n{option-illegal}{#1}},
    y/.code={\sseq@error@n{option-illegal}{#1}},
    z/.code={\sseq@error@n{option-illegal}{#1}},
    xshift/.sseq @ first pass code=\sseq@tikz@firstpass@checkshifttransform{x}{#1},
    yshift/.sseq @ first pass code=\sseq@tikz@firstpass@checkshifttransform{y}{#1},
    xshift/.append code=\sseq@options@scopeshiftsecondpass{\sseq@tikz@shifttransform{x}{#1}},%% Note special scopeshiftsecondpass command
    yshift/.append code=\sseq@options@scopeshiftsecondpass{\sseq@tikz@shifttransform{y}{#1}},%%
    xshift/.value required,
    yshift/.value required,
    shift/.sseq @ transform, shift/.value required,
    shift only/.sseq @ transform,
    xscale/.sseq @ transform, xscale/.value required,
    yscale/.sseq @ transform, yscale/.value required,
    scale/.sseq @ transform,  scale/.value required,
    scale around/.sseq @ transform, scale around/.value required,
    xslant/.sseq @ transform, xslant/.value required,
    yslant/.sseq @ transform, yslant/.value required,
    rotate/.sseq @ transform, rotate/.value required,
    rotate around/.sseq @ transform, rotate around/.value required,
    cm/.sseq @ transform, cm/.value required,
    reset cm/.sseq @ transform, reset cm/.value forbidden,
    transform canvas/.sseq @ transform, transform canvas/.value required
}

\def\sseq@shift@firstpassprocess#1{\sseq@shift@firstpassprocess@#1\sseq@nil}
\def\sseq@shift@firstpassprocess@#1#2\sseq@nil{\ifx#1\pgfpointxy\sseq@shift@firstpassprocess@@#2\else\sseq@standardparse\fi}
\def\sseq@shift@firstpassprocess@@#1#2{
    \sseq@tikz@firstpass@checkshifttransform{x}{#1}
    \sseq@tikz@firstpass@checkshifttransform{y}{#2}
}

\def\sseq@shift@secondpassprocess#1{\sseq@shift@secondpassprocess@#1\sseq@nil}
\def\sseq@shift@secondpassprocess@#1#2\sseq@nil{\ifx#1\pgfpointxy\sseq@shift@secondpassprocess@@#2\else\sseq@standardparse \pgftransformshift{#1#2}\fi}
\def\sseq@shift@secondpassprocess@@#1#2{
    \sseq@tikz@shifttransform{x}{#1}
    \sseq@tikz@shifttransform{y}{#2}
}

% special second pass handler for scopes to prevent shifts from causing out of memory style errors,
% see the longer comment at the definition of sseq@scope in main.code.tex
\def\sseq@options@scopeshiftsecondpass{\sseq@options@secondpass}

\sseq@pgfqkeysdirlist{/sseqpages/scope,/sseqpages/tikz primitives,/sseqpages,/sseqpages/label,/sseqpages/class/label}{
    align/.sseq @ second pass code=\tikzset{align/#1},
    align/.sseq @ first pass append code={\sseq@needstikztrue\def\sseq@classquotes@smugglehook@inner{\sseq@needstikztrue\tikzset{align/#1}}}
}

\sseq@pgfqkeysdirlist{/sseqpages/scope,/sseqpages/tikz primitives}{
    shift/.code={
        \sseq@options@firstpass{\tikz@scan@one@point\sseq@shift@firstpassprocess#1\relax}
        \sseq@options@secondpass{\tikz@scan@one@point\sseq@shift@secondpassprocess#1\relax} % TODO: maybe try to save the work of processing the point twice?
    },
    xshift/.sseq @ first pass code=\sseq@tikz@firstpass@checkshifttransform{x}{#1},
    yshift/.sseq @ first pass code=\sseq@tikz@firstpass@checkshifttransform{y}{#1},
    xshift/.append code=\sseq@options@scopeshiftsecondpass{\sseq@tikz@shifttransform{x}{#1}},%% Note special scopeshiftsecondpass command
    yshift/.append code=\sseq@options@scopeshiftsecondpass{\sseq@tikz@shifttransform{y}{#1}},%%
    rotate around/.sseq @ second pass code={\sseq@standardparse\sseq@pgfkeysdocommand{rotate}{#1}},
    rotate/.sseq @ stdparse do transform, rotate/.value required,
    xslant/.sseq @ stdparse do transform, xslant/.value required,
    yslant/.sseq @ stdparse do transform, yslant/.value required,
    xmirror/.sseq @ stdparse do transform={-1}, xmirror/.value forbidden,
    ymirror/.sseq @ stdparse do transform={-1}, ymirror/.value forbidden,
    xscale/.sseq @ stdparse do transform, xscale/.value required,
    yscale/.sseq @ stdparse do transform, yscale/.value required,
     scale/.sseq @ stdparse do transform,  scale/.value required,
    scale around/.sseq @ stdparse do transform, scale around/.value required
}

%%% Error checking for nonglobal transformations:

\def\sseq@firstpass@checkshifttransform#1#2{%
    \pgfmathparse{#2}%
    \sseq@ifpgfmathresultisunitlessint{%
        \@xp\pgfmathint\@xp{\pgfmathresult}%
        \@xp\advance\csname sseq@#1\endcsname\pgfmathresult\relax
    }{\sseq@error@n{illegal-transform}{#1shift}}%
}
\let\sseq@firstpass@shifttransform\sseq@firstpass@checkshifttransform

\def\sseq@checkshifttransform#1#2{%
    \pgfmathparse{(#2)}%
    \sseq@ifpgfmathresultisunitlessint{%
        \@xp\pgfmathint\@xp{\pgfmathresult}%
        \@xp\advance\csname sseq@#1\endcsname\pgfmathresult\relax
        \sseq@pgfkeysdocommand{#1shift}{#2cm}
    }{\sseq@error@n{illegal-transform}{#1shift}}%
}

\def\sseq@okayshifttransform#1#2{%
    \pgfmathparse{#2}
    \ifpgfmathunitsdeclared
        \sseq@pgfkeysdocommand{#1shift}{#2}
    \else
        \sseq@pgfkeysdocommand{#1shift}{#2cm}
    \fi
}

\def\sseq@tikz@firstpass@checkshifttransform#1#2{%
    \pgfmathparse{#2}%
    \sseq@ifpgfmathresultisunitlessint{%
        \@xp\pgfmathint\@xp{\pgfmathresult}%
        \@xp\advance\csname sseq@\sseq@getfirstchar#1\sseq@nil\endcsname\pgfmathresult\relax
    }{\sseq@standardparse}%
}

\def\sseq@tikz@shifttransform#1#2{%
    \pgfmathparse{#2}%
    \sseq@ifpgfmathresultisunitlessint{%
        \@xp\pgfmathint\@xp{\pgfmathresult}%
        \@xp\advance\csname sseq@\sseq@getfirstchar#1\sseq@nil\endcsname\pgfmathresult\relax
        \sseq@pgfkeysdocommand{#1shift}{#2cm*\csname sseq@#1scale\endcsname}
    }{
        \sseq@standardparse
        \ifpgfmathunitsdeclared
            \sseq@pgfkeysdocommand{#1shift}{#2}
        \else
            \sseq@pgfkeysdocommand{#1shift}{#2cm}
        \fi
    }%
}

\def\sseq@ifpgfmathresultisunitlessint{\ifpgfmathunitsdeclared\@xp\@secondoftwo\else\sseq@ifpgfmathresultisint{\@xp\@firstoftwo}{\@xp\@secondoftwo}\fi}
\def\sseq@ifpgfmathresultisint{\@xp\sseq@ifpgfmathresultisint@\pgfmathresult..\sseq@nil}
\def\sseq@ifpgfmathresultisint@#1.#2.#3\sseq@nil{\ifx\sseq@nil#2\sseq@nil\@xp\@firstoftwo\else \ifnum#2=\z@ \@xp\@xp\@xp\@firstoftwo\else \@xp\@xp\@xp\@secondoftwo\fi\fi}

%%%% Options for Tikz primitives

\def\sseq@pageconstraint{1}
\def\sseq@pageconstraint@true{1}

\sseq@pgfqkeysdirlist{/sseqpages/scope,/sseqpages/tikz primitives}{
    page constraint/.code={
        \sseq@updatepageconstraint{#1}{&&}
    },
    page constraint or/.code={
        \sseq@updatepageconstraint{#1}{||}
    },
    page/.code={\sseq@getnumrange\temp{#1}\sseq@updatepageconstraintrange},
    set page constraint/.store in=\sseq@pageconstraint,
    background/.code={\sseq@standardparse\sseq@setsavedpaths{background}\pgfkeysalso{standard parse}},
}

\sseq@pgfqkeysdirlist{/sseqpages/scope,/sseqpages/tikz primitives,/sseqpages/fit}{
    class style/.code={\sseq@d@addto@options{\sseq@thesseqstyle\sseq@theclassstyle\the\sseq@scope@toks}},
    cycle style/.forward to=/sseqpages/tikz primitives/class style,
    permanent class style/.code={\sseq@d@addto@options{\sseq@thesseqstyle\sseq@theclassstyle\sseq@thepermanentcyclestyle\the\sseq@scope@toks}},
    permanent cycle style/.forward to=/sseqpages/tikz primitives/permanent class style,
    transient class style/.code={\sseq@d@addto@options{\sseq@thesseqstyle\sseq@theclassstyle\sseq@thetransientcyclestyle\the\sseq@scope@toks}},
    transient cycle style/.forward to=/sseqpages/tikz primitives/transient class style,
    this page class style/.code={\sseq@d@addto@options{\sseq@thesseqstyle\sseq@theclassstyle\sseq@thetransientcyclestyle\sseq@thethispagecyclestyle\the\sseq@scope@toks}},
    this page cycle style/.forward to=/sseqpages/tikz primitives/this page class style,
    differential style/.code={\sseq@d@addto@options{\sseq@thesseqstyle\sseq@theedgestyle\sseq@thedifferentialstyle\the\sseq@scope@toks}},
    struct line style/.code={\sseq@d@addto@options{\sseq@thesseqstyle\sseq@theedgestyle\sseq@thestructlinestyle\the\sseq@scope@toks}},
    tikz primitive style/.code={\sseq@d@addto@options{\sseq@thesseqstyle\sseq@thetikzprimitivestyle\the\sseq@scope@toks}}
    %this page struct line style/.sseq @ first pass code={\sseq@d@addto@options{\sseq@thesseqstyle\sseq@theedgestyle\sseq@thestructlinestyle\the\sseq@scope@toks}}
}
